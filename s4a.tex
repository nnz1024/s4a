\documentclass[10pt,oneside,a4paper]{article}
\usepackage{cmap} % Copy-paste из PDF без проблем с кодировкой
\usepackage[utf8]{inputenc}
\usepackage[english,russian]{babel} % Русские переносы и проч.
\usepackage{graphicx,color}
\usepackage[T1,T2A]{fontenc}
\usepackage{indentfirst} % Отступ в первом абзаце главы 
\usepackage{fancyvrb} % Продвинутые листинги и in-line commands
% listings для данной ситуации, имхо, избыточен
\usepackage{pdflscape} % Внимание! При выводе в DVI выборочный
% поворот страниц работать не будет, хотя текст будет повернут.
\usepackage[colorlinks,unicode,urlcolor=blue]{hyperref}
% Заполняем поля PDF уже со включенной опцией unicode
\hypersetup{pdftitle={systemd для администраторов},%
pdfauthor={Lennart Poettering, Sergey Ptashnick}}
% Несколько сокращений
\newcommand{\sectiona}[1]{\section*{#1}\addcontentsline{toc}{section}{#1}}
\newcommand{\hreftt}[2]{\href{#1}{\texttt{#2}}}
\newcommand{\llangl}{\reflectbox{\rotatebox[origin=c]{270}{$\neg$}}}
% Настройка макета страницы
\setlength{\hoffset}{-1.5cm}
\addtolength{\textwidth}{2cm}
\setlength{\voffset}{-2cm}
\addtolength{\textheight}{3cm}
\addtolength{\footskip}{5pt}
% Настройка форматирования in-line commands
\DefineShortVerb{\+}
\VerbatimFootnotes
% И листингов
\definecolor{gray}{gray}{0.75}
\fvset{frame=leftline,rulecolor=\color{gray},framerule=1mm}
% Запрет висячих строк
\clubpenalty=10000
\widowpenalty=10000

\begin{document}
\sloppy
\title{systemd для администраторов}
\author{Lennart P\"{o}ttering (автор)\thanks{Первоисточник (на английском
языке) опубликован на сайте автора: \url{http://0pointer.de/blog/projects}}\\%
Сергей Пташник (русский перевод)\thanks{Актуальная версия перевода
доступна на портале OpenNet: \url{http://wiki.opennet.ru/Systemd}}\\%
\small Данный документ доступен на условиях лицензии
\href{http://creativecommons.org/licenses/by-sa/3.0/legalcode}{CC-BY-SA}}
\maketitle
\tableofcontents%\newpage
\sectiona{Предисловие автора}
Многие из вас, наверное, уже знают, что
\href{http://www.freedesktop.org/wiki/Software/systemd}{systemd}~--- это новая
система инициализации дистрибутива Fedora, начиная с Fedora~14\footnote{Прим.
перев.: к сожалению, разработчики Fedora приняли решение оставить в Fedora~14 в
качестве системы инициализации по умолчанию upstart, однако systemd все равно
включен в этот релиз и может быть использован в качестве альтернативной системы
инициализации. Окончательный переход на systemd произошел лишь в Fedora~15.}.
Помимо Fedora, systemd также поддерживает и другие дистрибутивы, в частности,
\href{http://en.opensuse.org/SDB:Systemd}{OpenSUSE}\footnote{Прим. перев.:
сейчас systemd поддерживается практически во всех популярных дистрибутивах для
настольных систем.}.  systemd предоставляет администраторам целый ряд новых
возможностей, значительно упрощающих процесс обслуживания системы. Эта статья
является первой в серии публикаций, планируемых в ближайшие месяцы. В каждой из
этих статей я попытаюсь рассказать об очередной новой возможности systemd.
Большинство этих возможностей можно описать легко и просто, и подобные статьи
должны быть интересны довольно широкой аудитории.  Однако, время от времени мы
будем рассматривать ключевые новшества systemd, что может потребовать несколько
более подробного изложения. 
\begin{flushright}
	Lennart P\"{o}ttering, 23 августа 2010~г.
\end{flushright}

\section{Контроль процесса загрузки}
Как правило, во время загрузки Linux по экрану быстро пробегает огромное
количество различных сообщений. Так как мы интенсивно работаем над
параллелизацией и ускорением процесса загрузки, с каждой новой версией
systemd эти сообщения будут пробегать все быстрее и быстрее, вследствие чего,
читать их будет все труднее. К тому же, многие пользователи применяют
графические оболочки загрузки (например, Plymouth), полностью скрывающие эти
сообщения. Тем не~менее, информация, которую несут эти сообщения, была и
остается чрезвычайно важной~--- они показывают, успешно ли запустилась каждая
служба, или попытка ее запуска закончилась ошибкой (зеленое
\texttt{[~\textcolor{green}{OK}~]} или красное
\texttt{[~\textcolor{red}{FAILED}~]} соответственно). Итак, с ростом скорости
загрузки систем, возникает неприятная ситуация: информация о результатах
запуска служб бывает очень важна, а просматривать ее все тяжелее. systemd
предлагает выход из этой ситуации: он отслеживает и запоминает факты успешного
или неудачного запуска служб на этапе загрузки, а также сбои служб во время
работы. К таким случая относятся выходы с ненулевым кодом, ошибки
сегментирования и т.п. Введя +systemctl status+ в своей командной оболочке, вы
можете ознакомиться с состоянием всех служб, как <<родных>> (native) для
systemd, так и классических SysV/LSB служб, поддерживаемых в целях
совместимости: 

\begin{landscape}
\begin{Verbatim}[fontsize=\small]
[root@lambda] ~# systemctl
UNIT                                          LOAD   ACTIVE       SUB        JOB       DESCRIPTION
dev-hugepages.automount                       loaded active       running              Huge Pages File System Automount Point
dev-mqueue.automount                          loaded active       running              POSIX Message Queue File System Automount Point
proc-sys-fs-binfmt_misc.automount             loaded active       waiting              Arbitrary Executable File Formats File System Automount Point
sys-kernel-debug.automount                    loaded active       waiting              Debug File System Automount Point
sys-kernel-security.automount                 loaded active       waiting              Security File System Automount Point
sys-devices-pc...0000:02:00.0-net-eth0.device loaded active       plugged              82573L Gigabit Ethernet Controller
[...]
sys-devices-virtual-tty-tty9.device           loaded active       plugged              /sys/devices/virtual/tty/tty9
-.mount                                       loaded active       mounted              /
boot.mount                                    loaded active       mounted              /boot
dev-hugepages.mount                           loaded active       mounted              Huge Pages File System
dev-mqueue.mount                              loaded active       mounted              POSIX Message Queue File System
home.mount                                    loaded active       mounted              /home
proc-sys-fs-binfmt_misc.mount                 loaded active       mounted              Arbitrary Executable File Formats File System
abrtd.service                                 loaded active       running              ABRT Automated Bug Reporting Tool
accounts-daemon.service                       loaded active       running              Accounts Service
acpid.service                                 loaded active       running              ACPI Event Daemon
atd.service                                   loaded active       running              Execution Queue Daemon
auditd.service                                loaded active       running              Security Auditing Service
avahi-daemon.service                          loaded active       running              Avahi mDNS/DNS-SD Stack
bluetooth.service                             loaded active       running              Bluetooth Manager
console-kit-daemon.service                    loaded active       running              Console Manager
cpuspeed.service                              loaded active       exited               LSB: processor frequency scaling support
crond.service                                 loaded active       running              Command Scheduler
cups.service                                  loaded active       running              CUPS Printing Service
dbus.service                                  loaded active       running              D-Bus System Message Bus
getty@tty2.service                            loaded active       running              Getty on tty2
getty@tty3.service                            loaded active       running              Getty on tty3
getty@tty4.service                            loaded active       running              Getty on tty4
getty@tty5.service                            loaded active       running              Getty on tty5
getty@tty6.service                            loaded active       running              Getty on tty6
haldaemon.service                             loaded active       running              Hardware Manager
hdapsd@sda.service                            loaded active       running              sda shock protection daemon
irqbalance.service                            loaded active       running              LSB: start and stop irqbalance daemon
iscsi.service                                 loaded active       exited               LSB: Starts and stops login and scanning of iSCSI devices.
iscsid.service                                loaded active       exited               LSB: Starts and stops login iSCSI daemon.
livesys-late.service                          loaded active       exited               LSB: Late init script for live image.
livesys.service                               loaded active       exited               LSB: Init script for live image.
lvm2-monitor.service                          loaded active       exited               LSB: Monitoring of LVM2 mirrors, snapshots etc. using dmeventd or progress polling
mdmonitor.service                             loaded active       running              LSB: Start and stop the MD software RAID monitor
modem-manager.service                         loaded active       running              Modem Manager
netfs.service                                 loaded active       exited               LSB: Mount and unmount network filesystems.
NetworkManager.service                        loaded active       running              Network Manager
ntpd.service                                  loaded maintenance  maintenance          Network Time Service
polkitd.service                               loaded active       running              Policy Manager
prefdm.service                                loaded active       running              Display Manager
rc-local.service                              loaded active       exited               /etc/rc.local Compatibility
rpcbind.service                               loaded active       running              RPC Portmapper Service
rsyslog.service                               loaded active       running              System Logging Service
rtkit-daemon.service                          loaded active       running              RealtimeKit Scheduling Policy Service
sendmail.service                              loaded active       running              LSB: start and stop sendmail
sshd@172.31.0.53:22-172.31.0.4:36368.service  loaded active       running              SSH Per-Connection Server
sysinit.service                               loaded active       running              System Initialization
systemd-logger.service                        loaded active       running              systemd Logging Daemon
udev-post.service                             loaded active       exited               LSB: Moves the generated persistent udev rules to /etc/udev/rules.d
udisks.service                                loaded active       running              Disk Manager
upowerd.service                               loaded active       running              Power Manager
wpa_supplicant.service                        loaded active       running              Wi-Fi Security Service
avahi-daemon.socket                           loaded active       listening            Avahi mDNS/DNS-SD Stack Activation Socket
cups.socket                                   loaded active       listening            CUPS Printing Service Sockets
dbus.socket                                   loaded active       running              dbus.socket
rpcbind.socket                                loaded active       listening            RPC Portmapper Socket
sshd.socket                                   loaded active       listening            sshd.socket
systemd-initctl.socket                        loaded active       listening            systemd /dev/initctl Compatibility Socket
systemd-logger.socket                         loaded active       running              systemd Logging Socket
systemd-shutdownd.socket                      loaded active       listening            systemd Delayed Shutdown Socket
dev-disk-by\x1...x1db22a\x1d870f1adf2732.swap loaded active       active               /dev/disk/by-uuid/fd626ef7-34a4-4958-b22a-870f1adf2732
basic.target                                  loaded active       active               Basic System
bluetooth.target                              loaded active       active               Bluetooth
dbus.target                                   loaded active       active               D-Bus
getty.target                                  loaded active       active               Login Prompts
graphical.target                              loaded active       active               Graphical Interface
local-fs.target                               loaded active       active               Local File Systems
multi-user.target                             loaded active       active               Multi-User
network.target                                loaded active       active               Network
remote-fs.target                              loaded active       active               Remote File Systems
sockets.target                                loaded active       active               Sockets
swap.target                                   loaded active       active               Swap
sysinit.target                                loaded active       active               System Initialization

LOAD   = Reflects whether the unit definition was properly loaded.
ACTIVE = The high-level unit activation state, i.e. generalization of SUB.
SUB    = The low-level unit activation state, values depend on unit type.
JOB    = Pending job for the unit.

221 units listed. Pass --all to see inactive units, too.
[root@lambda] ~#
\end{Verbatim}
(Листинг был сокращен за счет удаления строк, не~относящихся к теме статьи.)
\end{landscape}

Обратите внимание на графу ACTIVE, в которой отображается обобщенный статус
службы (или любого другого юнита systemd: устройства, сокета, точки
монтирования~--- их мы рассмотрим подробнее в последующих статьях). Основными
значениями обобщенного статуса являются active (служба выполняется) и inactive
(служба не~была запущена). Также существуют и другие статусы. Например,
внимательно посмотрев на листинг выше, вы можете заметить, что служба ntpd
(сервер точного времени) находится в состоянии, обозначенном как maintenance.
Чтобы узнать, что же произошло с ntpd, воспользуемся командой
+systemctl status+: 
\begin{Verbatim}[commandchars=\\\{\}]
[root@lambda] ~# systemctl status ntpd.service
ntpd.service - Network Time Service
	  Loaded: loaded (/etc/systemd/system/ntpd.service)
	  Active: \textcolor{red}{maintenance}
	    Main: 953 (code=exited, status=255)
	  CGroup: name=systemd:/systemd-1/ntpd.service
[root@lambda] ~#
\end{Verbatim}

systemd сообщает нам, что ntpd был запущен (с идентификатором процесса 953) и
аварийно завершил работу (с кодом выхода 255).

В последующих версиях systemd, мы планируем добавить возможность вызова в
таких ситуациях ABRT (Automated Bug Report Tool), но для этого необходима
поддержка со стороны самого ABRT. Соответствующий запрос уже
\href{https://bugzilla.redhat.com/show_bug.cgi?id=622773}{направлен} его
разработчикам, однако пока не~встретил среди них поддержки.

Резюме: использование +systemctl+ и +systemctl status+ является современной,
более удобной и эффективной альтернативой разглядыванию быстро пробегающих по
экрану сообщений в классическом SysV. +systemctl status+ дает возможность
получить развернутую информацию о характере ошибки и, кроме того, в отличие
от сообщений SysV, показывает не~только ошибки при запуске, но и ошибки,
возникшие во время исполнения службы. 

\section{О службах и процессах}
В большинстве современных Linux-систем количество одновременно работающих
процессов обычно весьма значительно. Понять, откуда взялся и что делает тот
или иной процесс, становится все сложнее и сложнее. Многие службы используют
сразу несколько рабочих процессов, и это отнюдь не~всегда можно легко
распознать по выводу команды +ps+. Встречаются еще более сложные ситуации,
когда демон запускает сторонние процессы~--- например, веб-сервер выполняет
CGI-программы, а демон cron~--- команды, предписанные ему в crontab.

Немного помочь в решении этой проблемы может древовидная иерархия процессов,
отображаемая по команде +ps xaf+. Именно <<немного помочь>>, а не~решить
полностью. В частности, процессы, родители которых умирают раньше их самих,
становят потомками PID~1 (процесса init), что сразу затрудняет процесс
выяснения их происхождения. Кроме того, процесс может избавиться от связи с
родителем через две последовательные операции +fork()+ (в целом, эта возможность
признается нужной и полезной, и является частью используемого в Unix подхода
к разработке демонов). Также, не~будем забывать, что процесс легко может
изменить свое имя посредством +PR_SETNAME+, или задав значение
+argv[0]+, что также усложняет процесс его опознания\footnote{Прим.
перев.: стоит отметить, что перечисленные ситуации могут возникнуть не~только
вследствие ошибок в коде и/или конфигурации программ, но и в результате злого
умысла. Например, очень часто встречается ситуация, когда установленный на
взломанном сервере процесс-бэкдор маскируется под нормального демона, меняя
себе имя, скажем, на httpd.}.

systemd предлагает простой путь для решения обсуждаемой задачи. Запуская
очередной новый процесс, systemd помещает его в отдельную контрольную группу
с соответствующим именем. Контрольные группы Linux предоставляют очень
удобный инструмент для иерархической структуризации процессов: когда
какой-либо процесс порождает потомка, этот потомок автоматически включается в
ту же группу, что и родитель. При этом, что очень важно, непривилегированные
процессы не~могут изменить свое положение в этой иерархии. Таким образом,
контрольные группы позволяют точно установить происхождение конкретного
процесса, вне зависимости от того, сколько раз он форкался и переименовывал
себя~--- имя его контрольной группы невозможно спрятать или изменить. Кроме
того, при штатном завершении родительской службы, будут завершены и все
порожденные ею процессы, как бы они ни~пытались сбежать. С systemd уже
невозможна ситуация, когда после остановки web-сервера, некорректно
форкнувшийся CGI-процесс продолжает исполняться вплоть до последних секунд
работы системы.

В этой статье мы рассмотрим две простых команды, которые позволят вам
наглядно оценить схему взаимоотношений systemd и порожденных им процессов.
Первая из этих команд~--- все та же +ps+, однако на этот раз в ее параметры
добавлено указание выводить сведения по контрольным группам, а также другую
интересную информацию: 

\begin{landscape}
\begin{Verbatim}[fontsize=\small]
$ ps xawf -eo pid,user,cgroup,args
  PID USER     CGROUP                              COMMAND
    2 root     -                                   [kthreadd]
    3 root     -                                    \_ [ksoftirqd/0]
[...]
 4281 root     -                                    \_ [flush-8:0]
    1 root     name=systemd:/systemd-1             /sbin/init
  455 root     name=systemd:/systemd-1/sysinit.service /sbin/udevd -d
28188 root     name=systemd:/systemd-1/sysinit.service  \_ /sbin/udevd -d
28191 root     name=systemd:/systemd-1/sysinit.service  \_ /sbin/udevd -d
 1096 dbus     name=systemd:/systemd-1/dbus.service /bin/dbus-daemon --system --address=systemd: --nofork --systemd-activation
 1131 root     name=systemd:/systemd-1/auditd.service auditd
 1133 root     name=systemd:/systemd-1/auditd.service  \_ /sbin/audispd
 1135 root     name=systemd:/systemd-1/auditd.service      \_ /usr/sbin/sedispatch
 1171 root     name=systemd:/systemd-1/NetworkManager.service /usr/sbin/NetworkManager --no-daemon
 4028 root     name=systemd:/systemd-1/NetworkManager.service  \_ /sbin/dhclient -d -4 -sf /usr/libexec/nm-dhcp-client.action -pf /var/run/dhclient-wlan0.pid -lf /var/lib/dhclient/dhclient-7d32a784-ede9-4cf6-9ee3-60edc0bce5ff-wlan0.lease -
 1175 avahi    name=systemd:/systemd-1/avahi-daemon.service avahi-daemon: running [epsilon.local]
 1194 avahi    name=systemd:/systemd-1/avahi-daemon.service  \_ avahi-daemon: chroot helper
 1193 root     name=systemd:/systemd-1/rsyslog.service /sbin/rsyslogd -c 4
 1195 root     name=systemd:/systemd-1/cups.service cupsd -C /etc/cups/cupsd.conf
 1207 root     name=systemd:/systemd-1/mdmonitor.service mdadm --monitor --scan -f --pid-file=/var/run/mdadm/mdadm.pid
 1210 root     name=systemd:/systemd-1/irqbalance.service irqbalance
 1216 root     name=systemd:/systemd-1/dbus.service /usr/sbin/modem-manager
 1219 root     name=systemd:/systemd-1/dbus.service /usr/libexec/polkit-1/polkitd
 1242 root     name=systemd:/systemd-1/dbus.service /usr/sbin/wpa_supplicant -c /etc/wpa_supplicant/wpa_supplicant.conf -B -u -f /var/log/wpa_supplicant.log -P /var/run/wpa_supplicant.pid
 1249 68       name=systemd:/systemd-1/haldaemon.service hald
 1250 root     name=systemd:/systemd-1/haldaemon.service  \_ hald-runner
 1273 root     name=systemd:/systemd-1/haldaemon.service      \_ hald-addon-input: Listening on /dev/input/event3 /dev/input/event9 /dev/input/event1 /dev/input/event7 /dev/input/event2 /dev/input/event0 /dev/input/event8
 1275 root     name=systemd:/systemd-1/haldaemon.service      \_ /usr/libexec/hald-addon-rfkill-killswitch
 1284 root     name=systemd:/systemd-1/haldaemon.service      \_ /usr/libexec/hald-addon-leds
 1285 root     name=systemd:/systemd-1/haldaemon.service      \_ /usr/libexec/hald-addon-generic-backlight
 1287 68       name=systemd:/systemd-1/haldaemon.service      \_ /usr/libexec/hald-addon-acpi
 1317 root     name=systemd:/systemd-1/abrtd.service /usr/sbin/abrtd -d -s
 1332 root     name=systemd:/systemd-1/getty@.service/tty2 /sbin/mingetty tty2
 1339 root     name=systemd:/systemd-1/getty@.service/tty3 /sbin/mingetty tty3
 1342 root     name=systemd:/systemd-1/getty@.service/tty5 /sbin/mingetty tty5
 1343 root     name=systemd:/systemd-1/getty@.service/tty4 /sbin/mingetty tty4
 1344 root     name=systemd:/systemd-1/crond.service crond
 1346 root     name=systemd:/systemd-1/getty@.service/tty6 /sbin/mingetty tty6
 1362 root     name=systemd:/systemd-1/sshd.service /usr/sbin/sshd
 1376 root     name=systemd:/systemd-1/prefdm.service /usr/sbin/gdm-binary -nodaemon
 1391 root     name=systemd:/systemd-1/prefdm.service  \_ /usr/libexec/gdm-simple-slave --display-id /org/gnome/DisplayManager/Display1 --force-active-vt
 1394 root     name=systemd:/systemd-1/prefdm.service      \_ /usr/bin/Xorg :0 -nr -verbose -auth /var/run/gdm/auth-for-gdm-f2KUOh/database -nolisten tcp vt1
 1495 root     name=systemd:/user/lennart/1             \_ pam: gdm-password
 1521 lennart  name=systemd:/user/lennart/1                 \_ gnome-session
 1621 lennart  name=systemd:/user/lennart/1                     \_ metacity
 1635 lennart  name=systemd:/user/lennart/1                     \_ gnome-panel
 1638 lennart  name=systemd:/user/lennart/1                     \_ nautilus
 1640 lennart  name=systemd:/user/lennart/1                     \_ /usr/libexec/polkit-gnome-authentication-agent-1
 1641 lennart  name=systemd:/user/lennart/1                     \_ /usr/bin/seapplet
 1644 lennart  name=systemd:/user/lennart/1                     \_ gnome-volume-control-applet
 1646 lennart  name=systemd:/user/lennart/1                     \_ /usr/sbin/restorecond -u
 1652 lennart  name=systemd:/user/lennart/1                     \_ /usr/bin/devilspie
 1662 lennart  name=systemd:/user/lennart/1                     \_ nm-applet --sm-disable
 1664 lennart  name=systemd:/user/lennart/1                     \_ gnome-power-manager
 1665 lennart  name=systemd:/user/lennart/1                     \_ /usr/libexec/gdu-notification-daemon
 1670 lennart  name=systemd:/user/lennart/1                     \_ /usr/libexec/evolution/2.32/evolution-alarm-notify
 1672 lennart  name=systemd:/user/lennart/1                     \_ /usr/bin/python /usr/share/system-config-printer/applet.py
 1674 lennart  name=systemd:/user/lennart/1                     \_ /usr/lib64/deja-dup/deja-dup-monitor
 1675 lennart  name=systemd:/user/lennart/1                     \_ abrt-applet
 1677 lennart  name=systemd:/user/lennart/1                     \_ bluetooth-applet
 1678 lennart  name=systemd:/user/lennart/1                     \_ gpk-update-icon
 1408 root     name=systemd:/systemd-1/console-kit-daemon.service /usr/sbin/console-kit-daemon --no-daemon
 1419 gdm      name=systemd:/systemd-1/prefdm.service /usr/bin/dbus-launch --exit-with-session
 1453 root     name=systemd:/systemd-1/dbus.service /usr/libexec/upowerd
 1473 rtkit    name=systemd:/systemd-1/rtkit-daemon.service /usr/libexec/rtkit-daemon
 1496 root     name=systemd:/systemd-1/accounts-daemon.service /usr/libexec/accounts-daemon
 1499 root     name=systemd:/systemd-1/systemd-logger.service /lib/systemd/systemd-logger
 1511 lennart  name=systemd:/systemd-1/prefdm.service /usr/bin/gnome-keyring-daemon --daemonize --login
 1534 lennart  name=systemd:/user/lennart/1        dbus-launch --sh-syntax --exit-with-session
 1535 lennart  name=systemd:/user/lennart/1        /bin/dbus-daemon --fork --print-pid 5 --print-address 7 --session
 1603 lennart  name=systemd:/user/lennart/1        /usr/libexec/gconfd-2
 1612 lennart  name=systemd:/user/lennart/1        /usr/libexec/gnome-settings-daemon
 1615 lennart  name=systemd:/user/lennart/1        /usr/libexec/gvfsd
 1626 lennart  name=systemd:/user/lennart/1        /usr/libexec//gvfs-fuse-daemon /home/lennart/.gvfs
 1634 lennart  name=systemd:/user/lennart/1        /usr/bin/pulseaudio --start --log-target=syslog
 1649 lennart  name=systemd:/user/lennart/1         \_ /usr/libexec/pulse/gconf-helper
 1645 lennart  name=systemd:/user/lennart/1        /usr/libexec/bonobo-activation-server --ac-activate --ior-output-fd=24
 1668 lennart  name=systemd:/user/lennart/1        /usr/libexec/im-settings-daemon
 1701 lennart  name=systemd:/user/lennart/1        /usr/libexec/gvfs-gdu-volume-monitor
 1707 lennart  name=systemd:/user/lennart/1        /usr/bin/gnote --panel-applet --oaf-activate-iid=OAFIID:GnoteApplet_Factory --oaf-ior-fd=22
 1725 lennart  name=systemd:/user/lennart/1        /usr/libexec/clock-applet
 1727 lennart  name=systemd:/user/lennart/1        /usr/libexec/wnck-applet
 1729 lennart  name=systemd:/user/lennart/1        /usr/libexec/notification-area-applet
 1733 root     name=systemd:/systemd-1/dbus.service /usr/libexec/udisks-daemon
 1747 root     name=systemd:/systemd-1/dbus.service  \_ udisks-daemon: polling /dev/sr0
 1759 lennart  name=systemd:/user/lennart/1        gnome-screensaver
 1780 lennart  name=systemd:/user/lennart/1        /usr/libexec/gvfsd-trash --spawner :1.9 /org/gtk/gvfs/exec_spaw/0
 1864 lennart  name=systemd:/user/lennart/1        /usr/libexec/gvfs-afc-volume-monitor
 1874 lennart  name=systemd:/user/lennart/1        /usr/libexec/gconf-im-settings-daemon
 1903 lennart  name=systemd:/user/lennart/1        /usr/libexec/gvfsd-burn --spawner :1.9 /org/gtk/gvfs/exec_spaw/1
 1909 lennart  name=systemd:/user/lennart/1        gnome-terminal
 1913 lennart  name=systemd:/user/lennart/1         \_ gnome-pty-helper
 1914 lennart  name=systemd:/user/lennart/1         \_ bash
29231 lennart  name=systemd:/user/lennart/1         |   \_ ssh tango
 2221 lennart  name=systemd:/user/lennart/1         \_ bash
 4193 lennart  name=systemd:/user/lennart/1         |   \_ ssh tango
 2461 lennart  name=systemd:/user/lennart/1         \_ bash
29219 lennart  name=systemd:/user/lennart/1         |   \_ emacs systemd-for-admins-1.txt
15113 lennart  name=systemd:/user/lennart/1         \_ bash
27251 lennart  name=systemd:/user/lennart/1             \_ empathy
29504 lennart  name=systemd:/user/lennart/1             \_ ps xawf -eo pid,user,cgroup,args
 1968 lennart  name=systemd:/user/lennart/1        ssh-agent
 1994 lennart  name=systemd:/user/lennart/1        gpg-agent --daemon --write-env-file
18679 lennart  name=systemd:/user/lennart/1        /bin/sh /usr/lib64/firefox-3.6/run-mozilla.sh /usr/lib64/firefox-3.6/firefox
18741 lennart  name=systemd:/user/lennart/1         \_ /usr/lib64/firefox-3.6/firefox
28900 lennart  name=systemd:/user/lennart/1             \_ /usr/lib64/nspluginwrapper/npviewer.bin --plugin /usr/lib64/mozilla/plugins/libflashplayer.so --connection /org/wrapper/NSPlugins/libflashplayer.so/18741-6
 4016 root     name=systemd:/systemd-1/sysinit.service /usr/sbin/bluetoothd --udev
 4094 smmsp    name=systemd:/systemd-1/sendmail.service sendmail: Queue runner@01:00:00 for /var/spool/clientmqueue
 4096 root     name=systemd:/systemd-1/sendmail.service sendmail: accepting connections
 4112 ntp      name=systemd:/systemd-1/ntpd.service /usr/sbin/ntpd -n -u ntp:ntp -g
27262 lennart  name=systemd:/user/lennart/1        /usr/libexec/mission-control-5
27265 lennart  name=systemd:/user/lennart/1        /usr/libexec/telepathy-haze
27268 lennart  name=systemd:/user/lennart/1        /usr/libexec/telepathy-logger
27270 lennart  name=systemd:/user/lennart/1        /usr/libexec/dconf-service
27280 lennart  name=systemd:/user/lennart/1        /usr/libexec/notification-daemon
27284 lennart  name=systemd:/user/lennart/1        /usr/libexec/telepathy-gabble
27285 lennart  name=systemd:/user/lennart/1        /usr/libexec/telepathy-salut
27297 lennart  name=systemd:/user/lennart/1        /usr/libexec/geoclue-yahoo
\end{Verbatim}
(Данный листинг был сокращен за счет удаления из него строк, описывающих
потоки ядра, так как они никак не~относятся к обсуждаемой нами теме.)
\end{landscape}

Обратите внимание на третий столбец, показывающий имя контрольной группы,
которое systemd присваивает каждому процессу. Например, процесс +udev+
находится в группе +name=systemd:/systemd-1/sysinit.service+. В эту группу
входят процессы, порожденные службой +sysinit.service+, которая запускается
на ранней стадии загрузки.

Вы можете очень сильно упростить себе работу, если назначите для
вышеприведенной команды какой-нибудь простой и короткий псевдоним, например 
\begin{Verbatim}
alias psc='ps xawf -eo pid,user,cgroup,args'
\end{Verbatim}
---~теперь для получения исчерпывающей информации по процессам достаточно будет
нажать всего четыре клавиши.

Альтернативный способ получить ту же информацию~--- воспользоваться утилитой
+systemd-cgls+, входящей в комплект поставки systemd. Она отображает иерархию
контрольных групп в виде псевдографической диаграммы-дерева: 

\begin{landscape}
\begin{Verbatim}[fontsize=\small]
$ systemd-cgls
+    2 [kthreadd]
[...]
+ 4281 [flush-8:0]
+ user
| \ lennart
|   \ 1
|     +  1495 pam: gdm-password
|     +  1521 gnome-session
|     +  1534 dbus-launch --sh-syntax --exit-with-session
|     +  1535 /bin/dbus-daemon --fork --print-pid 5 --print-address 7 --session
|     +  1603 /usr/libexec/gconfd-2
|     +  1612 /usr/libexec/gnome-settings-daemon
|     +  1615 /ushr/libexec/gvfsd
|     +  1621 metacity
|     +  1626 /usr/libexec//gvfs-fuse-daemon /home/lennart/.gvfs
|     +  1634 /usr/bin/pulseaudio --start --log-target=syslog
|     +  1635 gnome-panel
|     +  1638 nautilus
|     +  1640 /usr/libexec/polkit-gnome-authentication-agent-1
|     +  1641 /usr/bin/seapplet
|     +  1644 gnome-volume-control-applet
|     +  1645 /usr/libexec/bonobo-activation-server --ac-activate --ior-output-fd=24
|     +  1646 /usr/sbin/restorecond -u
|     +  1649 /usr/libexec/pulse/gconf-helper
|     +  1652 /usr/bin/devilspie
|     +  1662 nm-applet --sm-disable
|     +  1664 gnome-power-manager
|     +  1665 /usr/libexec/gdu-notification-daemon
|     +  1668 /usr/libexec/im-settings-daemon
|     +  1670 /usr/libexec/evolution/2.32/evolution-alarm-notify
|     +  1672 /usr/bin/python /usr/share/system-config-printer/applet.py
|     +  1674 /usr/lib64/deja-dup/deja-dup-monitor
|     +  1675 abrt-applet
|     +  1677 bluetooth-applet
|     +  1678 gpk-update-icon
|     +  1701 /usr/libexec/gvfs-gdu-volume-monitor
|     +  1707 /usr/bin/gnote --panel-applet --oaf-activate-iid=OAFIID:GnoteApplet_Factory --oaf-ior-fd=22
|     +  1725 /usr/libexec/clock-applet
|     +  1727 /usr/libexec/wnck-applet
|     +  1729 /usr/libexec/notification-area-applet
|     +  1759 gnome-screensaver
|     +  1780 /usr/libexec/gvfsd-trash --spawner :1.9 /org/gtk/gvfs/exec_spaw/0
|     +  1864 /usr/libexec/gvfs-afc-volume-monitor
|     +  1874 /usr/libexec/gconf-im-settings-daemon
|     +  1882 /usr/libexec/gvfs-gphoto2-volume-monitor
|     +  1903 /usr/libexec/gvfsd-burn --spawner :1.9 /org/gtk/gvfs/exec_spaw/1
|     +  1909 gnome-terminal
|     +  1913 gnome-pty-helper
|     +  1914 bash
|     +  1968 ssh-agent
|     +  1994 gpg-agent --daemon --write-env-file
|     +  2221 bash
|     +  2461 bash
|     +  4193 ssh tango
|     + 15113 bash
|     + 18679 /bin/sh /usr/lib64/firefox-3.6/run-mozilla.sh /usr/lib64/firefox-3.6/firefox
|     + 18741 /usr/lib64/firefox-3.6/firefox
|     + 27251 empathy
|     + 27262 /usr/libexec/mission-control-5
|     + 27265 /usr/libexec/telepathy-haze
|     + 27268 /usr/libexec/telepathy-logger
|     + 27270 /usr/libexec/dconf-service
|     + 27280 /usr/libexec/notification-daemon
|     + 27284 /usr/libexec/telepathy-gabble
|     + 27285 /usr/libexec/telepathy-salut
|     + 27297 /usr/libexec/geoclue-yahoo
|     + 28900 /usr/lib64/nspluginwrapper/npviewer.bin --plugin /usr/lib64/mozilla/plugins/libflashplayer.so --connection /org/wrapper/NSPlugins/libflashplayer.so/18741-6
|     + 29219 emacs systemd-for-admins-1.txt
|     + 29231 ssh tango
|     \ 29519 systemd-cgls
\ systemd-1
  + 1 /sbin/init
  + ntpd.service
  | \ 4112 /usr/sbin/ntpd -n -u ntp:ntp -g
  + systemd-logger.service
  | \ 1499 /lib/systemd/systemd-logger
  + accounts-daemon.service
  | \ 1496 /usr/libexec/accounts-daemon
  + rtkit-daemon.service
  | \ 1473 /usr/libexec/rtkit-daemon
  + console-kit-daemon.service
  | \ 1408 /usr/sbin/console-kit-daemon --no-daemon
  + prefdm.service
  | + 1376 /usr/sbin/gdm-binary -nodaemon
  | + 1391 /usr/libexec/gdm-simple-slave --display-id /org/gnome/DisplayManager/Display1 --force-active-vt
  | + 1394 /usr/bin/Xorg :0 -nr -verbose -auth /var/run/gdm/auth-for-gdm-f2KUOh/database -nolisten tcp vt1
  | + 1419 /usr/bin/dbus-launch --exit-with-session
  | \ 1511 /usr/bin/gnome-keyring-daemon --daemonize --login
  + getty@.service
  | + tty6
  | | \ 1346 /sbin/mingetty tty6
  | + tty4
  | | \ 1343 /sbin/mingetty tty4
  | + tty5
  | | \ 1342 /sbin/mingetty tty5
  | + tty3
  | | \ 1339 /sbin/mingetty tty3
  | \ tty2
  |   \ 1332 /sbin/mingetty tty2
  + abrtd.service
  | \ 1317 /usr/sbin/abrtd -d -s
  + crond.service
  | \ 1344 crond
  + sshd.service
  | \ 1362 /usr/sbin/sshd
  + sendmail.service
  | + 4094 sendmail: Queue runner@01:00:00 for /var/spool/clientmqueue
  | \ 4096 sendmail: accepting connections
  + haldaemon.service
  | + 1249 hald
  | + 1250 hald-runner
  | + 1273 hald-addon-input: Listening on /dev/input/event3 /dev/input/event9 /dev/input/event1 /dev/input/event7 /dev/input/event2 /dev/input/event0 /dev/input/event8
  | + 1275 /usr/libexec/hald-addon-rfkill-killswitch
  | + 1284 /usr/libexec/hald-addon-leds
  | + 1285 /usr/libexec/hald-addon-generic-backlight
  | \ 1287 /usr/libexec/hald-addon-acpi
  + irqbalance.service
  | \ 1210 irqbalance
  + avahi-daemon.service
  | + 1175 avahi-daemon: running [epsilon.local]
  + NetworkManager.service
  | + 1171 /usr/sbin/NetworkManager --no-daemon
  | \ 4028 /sbin/dhclient -d -4 -sf /usr/libexec/nm-dhcp-client.action -pf /var/run/dhclient-wlan0.pid -lf /var/lib/dhclient/dhclient-7d32a784-ede9-4cf6-9ee3-60edc0bce5ff-wlan0.lease -cf /var/run/nm-dhclient-wlan0.conf wlan0
  + rsyslog.service
  | \ 1193 /sbin/rsyslogd -c 4
  + mdmonitor.service
  | \ 1207 mdadm --monitor --scan -f --pid-file=/var/run/mdadm/mdadm.pid
  + cups.service
  | \ 1195 cupsd -C /etc/cups/cupsd.conf
  + auditd.service
  | + 1131 auditd
  | + 1133 /sbin/audispd
  | \ 1135 /usr/sbin/sedispatch
  + dbus.service
  | +  1096 /bin/dbus-daemon --system --address=systemd: --nofork --systemd-activation
  | +  1216 /usr/sbin/modem-manager
  | +  1219 /usr/libexec/polkit-1/polkitd
  | +  1242 /usr/sbin/wpa_supplicant -c /etc/wpa_supplicant/wpa_supplicant.conf -B -u -f /var/log/wpa_supplicant.log -P /var/run/wpa_supplicant.pid
  | +  1453 /usr/libexec/upowerd
  | +  1733 /usr/libexec/udisks-daemon
  | +  1747 udisks-daemon: polling /dev/sr0
  | \ 29509 /usr/libexec/packagekitd
  + dev-mqueue.mount
  + dev-hugepages.mount
  \ sysinit.service
    +   455 /sbin/udevd -d
    +  4016 /usr/sbin/bluetoothd --udev
    + 28188 /sbin/udevd -d
    \ 28191 /sbin/udevd -d
\end{Verbatim}
(Как и предыдущий, этот листинг был сокращен за счет удаления перечня потоков
ядра.)
\end{landscape}

Как видно из листинга, данная команда наглядно показывает принадлежность
процессов к их контрольным группам, а следовательно, и к службам, так как
systemd именует группы в соответствии с названиями служб. Например, из
приведенного листинга нетрудно понять, что служба системного аудита
+auditd.service+ порождает три отдельных процесса: +auditd+,
+audispd+ и +sedispatch+.

Наиболее внимательные читатели, вероятно, уже заметили, что некоторые процессы
помещены в группу +/user/lennart/1+. Дело в том, что systemd занимается
отслеживанием и группировкой не~только процессов, относящихся к системным
службам, но и процессов, запущенных в рамках пользовательских сеансов. В
последующих статьях мы обсудим этот вопрос более подробно. 

\section{HOW-TO: преобразование SysV init-скрипта в systemd service-файл}

Традиционно, службы Unix и Linux (демоны) запускаются через SysV init-скрипты.
Эти скрипты пишутся на языке Bourne Shell (+/bin/sh+), располагаются в
специальном каталоге (обычно +/etc/rc.d/init.d/+) и вызываются с одним из
стандартных параметров (+start+, +stop+, +reload+ и т.п.)~--- таким образом
указывается действие, которое необходимо прозвести над службой (запустить,
остановить, заставить перечитать конфигурацию). При запуске службы такой
скрипт, как правило, вызывает бинарник демона, который, в свою очередь,
форкается, порождая фоновый процесс (т.е. демонизируется). Заметим, что
shell-скрипты, как правило, отличается низкой скоростью работы, излишней
подробностью изложения и крайней хрупкостью. Читать их, из-за изобилия
всевозможного вспомогательного и дополнительного кода, чрезвычайно тяжело.
Впрочем, нельзя не~упомянуть, что эти скрипты являются очень гибким
инструментом (ведь, по сути, это всего лишь код, который можно модифицировать
как угодно). С другой стороны, многие задачи, возникающие при работе со
службами, довольно тяжело решить средствами shell-скриптов. К таким
задачам относятся: организация параллельного исполнения, корректное
отслеживание процессов, конфигурирование различных параметров среды исполнения
процесса. systemd обеспечивает совместимость с init-скриптами, однако, с учетом
описанных выше их недостатков, более правильным решением будет использование
штатных service-файлов systemd для всех установленных в системе служб. Стоит
отметить что, в отличие от init-скриптов, которые часто приходится
модифицировать при переносе из одного дистрибутива в другой, один и тот же
service-файл будет работать в любом дистрибутиве, использующем systemd (а таких
дистрибутивов с каждым днем становится все больше и больше).  Далее мы вкратце
рассмотрим процесс преобразования SysV init-скрипта в service-файл systemd.
Вообще говоря, service-файл должен создаваться разработчиками каждого демона, и
включаться в комплект его поставки. Если вам удалось успешно создать
работоспособный service-файл для какого-либо демона, настоятельно рекомендуем
вам отправить этот файл разработчикам. Вопросы по полноценной интеграции
демонов с systemd, с максимальным использованием всех его возможностей, будут
рассмотрены в последующих статьях этого цикла, пока же ограничимся ссылкой на 
\href{http://0pointer.de/public/systemd-man/daemon.html}{страницу} официальной
документации. 

Итак, приступим. В качестве пример возьмем init-скрипт демона ABRT (Automatic
Bug Reporting Tool, службы, занимающейся сбором crash dump'ов). Исходный
скрипт (в варианте для дистрибутива Fedora) можно загрузить
\href{http://0pointer.de/public/abrtd}{здесь}.

Начнем с того, что прочитаем исходный скрипт (неожиданный ход, правда?) и
выделим полезную информацию из груды хлама. Практически у всех init-скриптов
б\'{о}льшая часть кода является чисто вспомогательной, и мало чем отличается от
одного скрипта к другому. Как правило, при создании новых скриптов этот код
просто копируется из уже существующих (разработка в стиле copy-paste). Итак,
в исследуемом скрипте нас интересует следующая информация: 

\begin{itemize}
	\item Строка описания службы: <<Daemon to detect crashing apps>>. Как
		нетрудно заметить, комментарии в заголовке скрипта весьма
		пространны и описывают не~сколько саму службу, сколько
		скрипт, ее запускающий. service-файлы systemd также включают
		описание, но оно относится исключительно к службе, а не~к
		service-файлу. 
	\item LSB-заголовок\footnote{LSB-заголовок~--- определенная в
		\href{http://refspecs.freestandards.org/LSB_3.1.1/LSB-Core-generic/LSB-Core-generic/initscrcomconv.html}{Linux
		Standard Base} схема записи метаданных о службах в блоках
		комментариев соответствующих init-скриптов. Изначально эта
		схема была введена именно для того, чтобы стандартизировать
		init-скрипты во всех дистрибутивах. Однако разработчики
		многих дистрибутивов не~считают нужным точно исполнять
		требования LSB, и поэтому формы представления метаданных в
		различных дистрибутивах могут отличаться. Вследствие этого,
		при переносе init-скрипта из одного дистрибутива в другой,
		скрипт приходится модифицировать. Например, демон пересылки
		почты при описании зависимостей может именоваться
		+MTA+ или +smtpdaemon+ (Fedora), +smtp+
		(openSUSE), +mail-transport-agent+ (Debian и Ubuntu),
		+mail-transfer-agent+. Таким образом, можно утверждать, что
		стандарт LSB не~справляется с возложенной на него задачей.},
		содержащий информацию о зависимостях.  systemd, базирующийся
		на идеях socket-активации, обычно не~требует явного описания
		зависимостей (либо требует самого минимального описания).
		Заметим, что основополагающие принципы systemd, включая
		socket-активацию, рассмотрены в статье
		\href{http://0pointer.de/blog/projects/systemd.html}{Rethinking
		PID~1}, в которой systemd был впервые представлен широкой
		публике. Ее русский перевод можно прочитать здесь:
		\href{http://tux-the-penguin.blogspot.com/2010/09/systemd.html}{часть~1},
		\href{http://tux-the-penguin.blogspot.com/2010/09/systemd-ii.html}{часть~2}.
		Возвращаясь к нашему примеру: в данном случае ценной
		информацией о зависимостях является только строка
		+Required-Start: $syslog+, сообщающая, что для работы
		abrtd требуется демон системного лога.  Информация о второй
		зависимости, +$local_fs+, является избыточной, так как
		systemd приступает к запуску служб уже после того, как все
		файловые системы готовы для работы. 
	\item Также, LSB-заголовок сообщает, что данная служба должна быть
		запущена на уровнях исполнения (runlevels) 3 (консольный
		многопользовательский) и 5 (графический
		многопользовательской).
	\item Исполняемый бинарник демона называется +/usr/sbin/abrtd+. 
\end{itemize}

Вот и вся полезная информация. Все остальное содержимое 115-строчного скрипта
является чисто вспомогательным кодом: операции синхронизации и упорядочивания
запуска (код, относящийся к lock-файлам), вывод информационных сообщений
(команды +echo+), разбор входных параметров (монструозный блок
+case+).

На основе приведенной выше информации, мы можем написать следующий
service-файл: 
\begin{Verbatim}
[Unit]
Description=Daemon to detect crashing apps
After=syslog.target

[Service]
ExecStart=/usr/sbin/abrtd
Type=forking

[Install]
WantedBy=multi-user.target
\end{Verbatim}

Рассмотрим этот файл поподробнее.

Секция +[Unit]+ содержит самую общую информацию о службе. Не~будем
забывать, что systemd управляет не~только службами, но и многими другими
объектами, в частности, устройствами, точками монтирования, таймерами и т.п.
Общее наименование всех этих объектов~--- юнит (unit). Одноименная секция
конфигурационного файла определяет наиболее общие свойства, которые могут
быть присущи любому юниту. В нашем случае это, во-первых, строка описания, и
во-вторых, указание, что данный юнит рекомендуется активировать после запуска
демона системного лога\footnote{Строго говоря, эту зависимость здесь
указывать не~нужно~--- в системах, в которых демон системного лога активируется
через сокет, данная зависимость является избыточной. Современные реализации
демона системного лога (например, rsyslog начиная с пятой версии)
поддерживают активацию через сокет. В системах, использующих такие
реализации, явное указание +After=syslog.target+ будет избыточным, так
как соответствующая функциональность поддерживается автоматически. Однако,
эту строчку стоит все-таки указать для обеспечения совместимости с системами,
использующими устаревшие реализации демона системного лога.}. Эта информация,
как мы помним, была указана в LSB-заголовке исходного init-скрипта. В нашем
конфигурационном файле мы указываем зависимость от демона системного лога при
помощи директивы +After+, указывающей на юнит +syslog.taget+. Это
специальный юнит, позволяющий ссылаться на любую реализацию демона системного
лога, независимо от используемой программы (например, rsyslog или syslog-ng)
и типа активации (как обычной службы или через log-сокет). Подробнее о таких
специальных юнитах можно почитать
\href{http://0pointer.de/public/systemd-man/systemd.special.html}{страницу}
официальной документации. Обратите внимание, что директива +After+, в
отсутствие директивы +Requires+, задает лишь порядок загрузки, но
не~задает жесткой зависимости. То есть, если при загрузке конфигурация
systemd будет предписывать запуск как демона системного лога, так и abrtd, то
сначала будет запущен демон системного лога, и только потом abrtd. Если же
конфигурация не~будет содержать явного указания запустить демон системного
лога, он не~будет запущен даже при запуске abrtd. И это поведение нас
полностью устраивает, так как abrtd прекрасно может обходиться и без демона
системного лога. В противном случае, мы могли бы воспользоваться директивой
+Requires+, задающей жесткую зависимость между юнитами.

Следующая секция, +[Service]+, содержит информацию о службе. Сюда включаются
настройки, относящие именно к службам, но не~к другим типам юнитов. В нашем
случае, таких настроек две: +ExecStart+, определяющая расположение бинарника
демона и аргументы, с которыми он будет вызван (в нашем случае они
отсутствуют), и +Type+, позволяющая задать метод, по которому systemd определит
окончание периода запуска службы. Традиционный для Unix метод демонизации
процесса, когда исходный процесс форкается, порождая демона, после чего
завершается, описывается типом +forking+ (как в нашем случае). Таким образом,
systemd считает службу запущенной с момента завершения работы исходного
процесса, и рассматривает в качестве основного процесса этой службы
порожденный им процесс-демон.

И наконец, третья секция, +[Install]+. Она содержит рекомендации по
установке конкретного юнита, указывающие, в каких ситуациях он должен быть
активирован.  В нашем случае, служба abrtd запускается при активации юнита
+multi-user.target+. Это специальный юнит, примерно соответствующий роли
третьего уровня исполнения классического SysV\footnote{В том контексте, в
котором он используется в большинстве дистрибутивов семейства Red Hat, а
именно, многопользовательский режим без запуска графической оболочки.}.
Директива +WantedBy+ никак не~влияет на уже работающую службу, но она
играет важную роль при выполнении команды +systemctl enable+, задавая, в каких
условиях должен активироваться устанавливаемый юнит. В нашем примере, служба
abrtd будет активироваться при переходе в состояние +multi-user.target+,
т.е., при каждой нормальной загрузке\footnote{Обратите внимание, что режим
графической загрузки в systemd (+graphical.target+, аналог runlevel 5
в SysV) является надстройкой над режимом многопользовательской консольной
загрузки (+multi-user.target+, аналог runlevel 3 в SysV). Таким
образом, все службы, запускаемые в режиме +multi-user.target+, будут
также запускаться и в режиме +graphical.target+.} (к <<ненормальным>>
можно отнести, например, загрузки в режиме +emergency.target+, который
является аналогом первого уровня исполнения в классической SysV).

Вот и все. Мы получили минимальный рабочий service-файл systemd. Чтобы
проверить его работоспособность, скопируем его в
+/etc/systemd/system/abrtd.service+, после чего командой
+systemctl daemon-reload+ уведомим systemd об изменении конфигурации.
Теперь нам остается только запустить нашу службу:
+systemctl start abrtd.service+.  Проверить состояние службы можно
командой +systemctl status abrtd.service+, а чтобы остановить ее, нужно
скомандовать +systemctl stop abrtd.service+. И наконец, команда
+systemctl enable abrtd.service+ выполнит установку service-файла,
обеспечив его активацию при каждой загрузке (аналог +chkconfig abrtd on+
в классическом SysV).

Приведенный выше service-файл является практический точным переводом
исходного init-скрипта, и он никак не~использует широкий спектр возможностей,
предоставляемых systemd. Ниже приведен немного улучшенный вариант этого же
файла: 

\begin{Verbatim}
[Unit]
Description=ABRT Automated Bug Reporting Tool
After=syslog.target

[Service]
Type=dbus
BusName=com.redhat.abrt
ExecStart=/usr/sbin/abrtd -d -s

[Install]
WantedBy=multi-user.target
\end{Verbatim}

Чем же новый вариант отличается от предыдущего? Ну, прежде всего, мы уточнили
описание службы. Однако, ключевым изменением является замена значения +Type+ с +forking+
на +dbus+ и связанные с ней изменения: добавление имени службы в шине D-Bus
(директива +BusName+) и задание дополнительных аргументов abrtd <<+-d -s+>>. Но
зачем вообще нужна эта замена? Каков ее практический смысл? Чтобы ответить на
этот вопрос, мы снова возвращаемся к демонизации. В ходе этой операции процесс
дважды форкается и отключается от всех терминалов. Это очень удобно при запуске
демона через скрипт, но в случае использования таких продвинутых систем
инициализации, как systemd, подобное поведение не~дает никаких преимуществ, но
вызывает неоправданные задержки. Даже если мы оставим в стороне вопрос скорости
загрузки, останется такой важный аспект, как отслеживание состояния служб.
systemd решает и эту задачу, контролируя работу службы и при необходимости
реагируя на различные события. Например, при неожиданном падении основного
процесса службы, systemd должен зарегистрировать идентификатор и код выхода
процесса, также, в зависимости от настроек, он может попытаться перезапустить
службу, либо активировать какой-либо заранее заданный юнит. Операция
демонизации несколько затрудняет решение этих задач, так как обычно довольно
сложно найти связь демонизированного процесса с исходным (собственно, смысл
демонизации как раз и сводится к уничтожению этой связи) и, соответственно, для
systemd сложнее определить, какой из порожденных в рамках данной службы
процессов является основным. Чтобы упростить для него решение этой задачи, мы и
воспользовались типом запуска +dbus+. Он подходит для всех служб, которые в
конце процесса инициализации регистрируют свое имя на шине D-Bus\footnote{В
настоящее время практически все службы дистрибутива Fedora после запуска
регистрируется на шине D-Bus.}. ABRTd относится к ним.  С новыми настройками,
systemd запустит процесс abrtd, который уже не~будет форкаться (согласно
указанным нами ключам <<+-d -s+>>), и в качестве момента окончания периода
запуска данной службы systemd будет рассматривать момент регистрации имени
+com.redhat.abrt+ на шине D-Bus. В этом случае основным для данной службы будет
считаться процесс, непосредственно порожденный systemd.  Таким образом, systemd
располагает удобным методом для определения момента окончания запуска службы, а
также может легко отслеживать ее состояние. 

Собственно, это все, что нужно было сделать. Мы получили простой
конфигурационный файл, в 10 строчках которого содержится больше полезной
информации, чем в 115 строках исходного init-скрипта. Добавляя в наш файл по
одной строчке, мы можем использовать различные полезные функции systemd,
создание аналога которых в традиционном init-скрипте потребовало бы
значительных усилий. Например, добавив строку +Restart=restart-always+, мы
приказываем systemd автоматически перезапускать службу после каждого ее
падения. Или, например, добавив +OOMScoreAdjust=-500+, мы попросим ядро сберечь
эту службу, даже если OOM Killer выйдет на тропу войны. А если мы добавим
строчку +CPUSchedulingPolicy=idle+, процесс abrtd будет работать только в те
моменты, когда система больше ничем не~занята, что позволит не~создавать помех
для процессов, активно использующих CPU. 

За более подробным описанием всех опций настройки, вы можете обратиться к
страницам рукводства
\href{http://0pointer.de/public/systemd-man/systemd.unit.html}{systemd.unit},
\href{http://0pointer.de/public/systemd-man/systemd.service.html}{systemd.service},
\href{http://0pointer.de/public/systemd-man/systemd.exec.html}{systemd.exec}. Полный
список доступных страниц можно просмотреть
\href{http://0pointer.de/public/systemd-man/}{здесь}.

Конечно, отнюдь не~все init-скрипты так же легко преобразовать в
service-файлы. Но, к счастью, <<проблемных>> скриптов не~так уж и много. 

\section{Убить демона}

Убить системного демона нетрудно, правда? Или\ldots{} все не~так просто?

Если ваш демон функционирует как один процесс, все действительно очень просто.
Вы командуете +killall rsyslogd+, и демон системного лога останавливается.
Впрочем, этот метод не~вполне корректен, так как он действует не~только на
самого демона, но и на другие процессы с тем же именем. Иногда подобное
поведение может привести к неприятным последствиям. Более правильным будет
использование pid-файла: +kill $(cat /var/run/syslogd.pid)$+. Вот, вроде
бы, и все, что вам нужно\ldots{} Или мы упускаем еще что-то?

Действительно, мы забываем одну простую вещь: существуют службы, такие, как
Apache, crond, atd, которые по роду служебной деятельности должны запускать
дочерние процессы. Это могут быть совершенно посторонние, указанные
пользователем программы (например, задачи cron/at, CGI-скрипты) или полноценные
серверные процессы (например, Apache workers). Когда вы убиваете основной
процесс, он может остановить все дочерние процессы. А может и не~остановить. В
самом деле, если служба функционирует в штатном режиме, ее обычно останавливают
командой +stop+. К прямому вызову +kill+ администратор, как правило,
прибегает только в аварийной ситуации, когда служба работает неправильно и
может не~среагировать на стандартную команду остановки. Таким образом, убив,
например, основной сервер Apache, вы можете получить от него в наследство
работающие CGI-скрипты, причем их родителем автоматически станет PID~1 (init),
так что установить их происхождение будет не~так-то просто. 

\href{http://www.freedesktop.org/wiki/Software/systemd}{systemd} спешит к нам
на помощь. Команда +systemctl kill+ позволит отправить сигнал всем
процессам, порожденным в рамках данной службы. Например: 

\begin{Verbatim}
# systemctl kill crond.service
\end{Verbatim}

Вы можете быть уверены, что всем процессам службы cron будет отправлен сигнал
+SIGTERM+. Разумеется, можно отправить и любой другой сигнал. Скажем, если ваши
дела совсем уж плохи, вы можете воспользоваться и +SIGKILL+: 

\begin{Verbatim}
# systemctl kill -s SIGKILL crond.service
\end{Verbatim}

После ввода этой команды, служба cron будет жестоко убита вместе со всеми ее
дочерними процессами, вне зависимости от того, сколько раз она форкалась, и
как бы она ни пыталась сбежать из-под нашего контроля при помощи двойного
форка или
\href{http://ru.wikipedia.org/wiki/Fork-%D0%B1%D0%BE%D0%BC%D0%B1%D0%B0}{форк-бомбардировки}%
\footnote{Прим. перев.: стоит особо отметить, что использование контрольных
групп не~только упрощает процесс уничтожения форк-бомб, но и значительно
уменьшает ущерб от работающей форк-бомбы. Так как systemd автоматически
помещает каждую службу и каждый пользовательский сеанс в свою контрольную
группу по ресурсу процессорного времени, запуск форк-бомбы одним
пользователем или службой не~создаст значительных проблем с отзывчивостью
системы у других пользователей и служб. Таким образом, в качестве основной
угрозы форк-бомбардировки остаются лишь возможности исчерпания памяти и
идентификаторов процессов (PID).}.

В некоторый случах возникает необходимость отправить сигнал именно основному
процессу службы. Например, используя +SIGHUP+, мы можем заставить демона
перечитать файлы конфигурации. Разумеется, передавать HUP вспомогательным процессам
в этом случае совершенно необязательно. Для решения подобной
задачи неплохо подойдет и классический метод с pid-файлом, однако у
systemd и на этот случай есть простое решение, избавляющее вас от
необходимости искать нужный файл: 

\begin{Verbatim}
# systemctl kill -s HUP --kill-who=main crond.service
\end{Verbatim}

Итак, что же принципиально новое привносит systemd в рутинный процесс
убийства демона? Прежде всего: впервые в истории Linux представлен способ
принудительной остановки службы, не~зависящий от того, насколько
добросовестно основной процесс службы выполняет свои обязательства по
остановке дочерних процессов. Как уже упоминалось выше, необходимость
отправить процессу +SIGTERM+ или +SIGKILL+ обычно возникает именно
в нештатной ситуации, когда вы уже не~можете быть уверены, что демон
корректно исполнит все свои обязанности.

После прочтения сказанного выше у вас может возникнуть вопрос: в чем разница
между +systemctl kill+ и +systemctl stop+? Отличие состоит в том,
что +kill+ просто отправляет сигнал заданному процессу, в то время как
+stop+ действует по <<официально>> определенному методу, вызывая команду,
определенную в параметре +ExecStop+ конфигурации службы. Обычно команды
+stop+ бывает вполне достаточно для остановки службы, и к +kill+
приходится прибегать только в крайних случаях, например, когда служба
<<зависла>> и не~реагирует на команды.

Кстати говоря, при использовании параметра <<+-s+>>, вы можете указывать
названия сигналов как с префиксом SIG, так и без него~--- оба варианта будут
работать.

В завершение стоит сказать, что для нас весьма интересным и неожиданным
оказался тот факт, что до появления systemd в Linux просто не~существовало
инструментов, позволяющих корректно отправить сигнал службе в целом, а
не~отдельному процессу.

\section{Три уровня выключения}

В \href{http://www.freedesktop.org/wiki/Software/systemd}{systemd} существует
три уровня (разновидности) действий, направленных на прекращение работы службы
(или любого другого юнита):

\begin{itemize}
	\fvset{gobble=3}
	\item Вы можете \emph{остановить} службу, то есть прекратить
		выполнение уже запущенных процессов службы. При этом
		сохраняется возможность ее последующего запуска, как ручного
		(через команду +systemctl start+), так и автоматического (при
		загрузке системы, при поступлении запроса через сокет или
		системную шину, при срабатывании таймера, при подключении
		соответствующего оборудования и т.д.).  Таким образом,
		остановка службы является временной мерой, не~дающей никаких
		гарантий на будущее.

		В качестве примера рассмотрим остановку службы NTPd
		(отвечающей за синхронизацию времени по сети):
		\begin{Verbatim}
			systemctl stop ntpd.service
		\end{Verbatim}

		Аналогом этой команды в классическом SysV init является
		\begin{Verbatim}
			service ntpd stop
		\end{Verbatim}

		Заметим, что в Fedora~15, использующей в качестве системы
		инициализации systemd, в целях обеспечения обратной
		совместимости допускается использование классических
		SysV-команд, и systemd будет корректно воспринимать их. 
		В~частности, вторая приведенная здесь команда будет эквивалентна
		первой.

	\item Вы можете \emph{отключить} службу, то есть отсоединить ее от всех
		триггеров активации. В результате служба уже не~будет
		автоматически запускаться ни~при загрузке системы, ни~при
		обращении к сокету или адресу на шине, ни~при подключении
		оборудования, и т.д. Но при этом сохраняется возможность
		<<ручного>> запуска службы (командой +systemctl start+).
		Обратите внимание, что при отключении уже запущенной службы, ее
		выполнение в текущем сеансе не~останавливается~--- это нужно
		сделать отдельно, иначе процессы службы будут работать до
		момента выключения системы (но при следующем включении,
		разумеется, уже не~запустятся).

		Рассмотрим отключение службы на примере все того же NTPd:
		\begin{Verbatim}
			systemctl disable ntpd.service
		\end{Verbatim}

		В классических SysV-системах аналогичная команда будет иметь
		вид
		\begin{Verbatim}
			chkconfig ntpd off
		\end{Verbatim}

		Как и в предыдущем случае, в Fedora~15 вторая из этих команд
		будет действовать аналогично первой.

		Довольно часто приходится сочетать действия отключения и
		остановки службы~--- такая комбинированная операция
		гарантирует, что уже исполняющиеся процессы службы будут
		прекращены, и служба больше не~будет запускаться автоматически
		(но может быть запущена вручную):
		\begin{Verbatim}
			systemctl disable ntpd.service
			systemctl stop ntpd.service
		\end{Verbatim}
		Подобное сочетание команд используется,
		например, при деинсталляции пакетов в Fedora.

		Обратите внимание, что отключение службы является перманентной
		мерой, и действует вплоть до явной отмены соответствующей
		командой. Перезагрузка системы не~отменяет отключения службы.

	\item Вы можете \emph{заблокировать} (замаскировать) службу. Действие
		этой операции аналогично отключению, но дает более сильный
		эффект. Если при отключении отменяется только возможность
		автоматического запуска службы, но сохраняется возможность
		ручного запуска, то при блокировке исключаются обе эти
		возможности. Отметим, что использование данной опции при
		непонимании принципов ее работы может привести к трудно
		диагностируемым ошибкам.

		Тем не~менее, рассмотрим пример блокировки все той же службы NTPd:
		\begin{Verbatim}
			ln -s /dev/null /etc/systemd/system/ntpd.service
			systemctl daemon-reload
		\end{Verbatim}

		Итак, блокировка сводится к созданию символьной ссылки
		с именем соответствующей службы, указывающей на +/dev/null+.
		После такой операции служба не~может быть запущена ни~вручную,
		ни~автоматически. Символьная ссылка создается в каталоге
		+/etc/systemd/system/+, а ее имя должно соответствовать имени
		файла описания службы из каталога +/lib/systemd/system/+ (в
		нашем случае +ntpd.service+).

		Заметим, что systemd читает файлы конфигурации из обоих этих
		каталогов, но файлы из +/etc+ (управляемые системным
		администратором) имеют приоритет над файлами из +/lib+ (которые
		управляются пакетным менеджером). Таким образом, создание
		символьной ссылки (или обычного файла)
		+/etc/systemd/system/ntpd.service+ предотвращает чтение
		штатного файла конфигурации +/lib/systemd/system/ntpd.service+.

		В выводе +systemctl status+ заблокированные службы отмечаются
		словом +masked+. Попытка запустить такие службы командой
		+systemctl start+ завершится ошибкой.

		В рамках классического SysV init, штатная реализация такой
		возможности отсутствует. Похожий эффект может быть
		достигнут с помощью <<костылей>>, например, путем добавления
		команды +exit 0+ в начало init-скрипта. Однако, подобные решения
		имеют ряд недостатков, например, потенциальная возможность
		конфликтов с пакетным менеджером (при очередном обновлении
		исправленный скрипт может быть просто затерт соответствующим
		файлом из пакета).

		Стоит отметить, что блокировка службы, как и ее отключение,
		является перманентной мерой\footnote{Прим. перев.: подробно
		описав принцип работы блокировки службы (юнита), автор забывает
		привести практические примеры ситуаций, когда эта возможность
		оказывается полезной.
		
		В частности, иногда бывает необходимо
		полностью предотвратить запуск службы в любой ситуации. При этом
		не~стоит забывать, что в post-install скриптах пакетного
		менеджера или, скажем, в~заданиях cron, вместо
		+systemctl try-restart+ (+service condrestart+) может быть
		ошибочно указано +systemctl restart+ (+service restart+), что
		является прямым указанием на запуск службы, если она еще
		не~запущена. Вследствие таких ошибок, отключенная служба может
		<<ожить>> в самый неподходящий момент.
		
		Другой пример~---
		ограничение возможностей непривилегированного пользователя при
		управлении системой. Даже если такому пользователю делегировано
		(через механизмы sudo или PolicyKit) право на использование
		+systemctl+, это еще не~означает, что он сможет запустить
		заблокированную службу~--- права на выполнение +rm+ (удаление
		блокирующей символьной ссылки) выдаются отдельно.}.
\end{itemize}

После прочтения изложенного выше, у читателя может возникнуть вопрос: как
отменить произведенные изменения? Что ж, ничего сложного тут нет:
+systemctl start+ отменяет действия +systemctl stop+, +systemctl enable+
отменяет действие +systemctl disable+, а +rm+ отменяет действие +ln+.

\section{Смена корня}

Практически все администраторы и разработчики рано или поздно встречаются с
\href{http://linux.die.net/man/1/chroot}{chroot-окружениями}. Системный вызов
+chroot()+ позволяет задать для определенного процесса (и его потомков) каталог,
который они будут рассматривать как корневой +/+, тем самым ограничивая для них
область видимости иерархии файловой системы отдельной ветвью. Большинство
применений chroot-окружений можно отнести к двум классам задач:
\begin{enumerate}
	\item Обеспечение безопасности. Потенциально уязвимый демон chroot'ится
		в отдельный каталог и, даже в случае успешной атаки, взломщик
		увидит лишь содержимое этого каталога, а не~всю файловую
		систему~--- он окажется в ловушке chroot'а.
	\item Подготовка и управление образом операционной системы при отладке,
		тестировании, компиляции, установке или восстановлении. При этом
		вся иерархия файловых систем гостевой ОС монтируется или
		создается в каталоге системы-хоста, и при запуске оболочки (или
		любого другого приложения) внутри этой иерархии, в качестве
		корня используется этот каталог. Система, которую <<видят>>
		такие программы, может сильно отличаться от ОС хоста. Например,
		это может быть другой дистрибутив, или даже другая аппаратная
		архитектура (запуск i386-гостя на x86\_64-хосте). Гостевая ОС
		не~может увидеть полного дерева каталогов ОС хоста.
\end{enumerate}

В системах, использующих классический SysV init, использовать chroot-окружения
сравнительно несложно. Например, чтобы запустить выбранного демона внутри дерева
каталогов гостевой ОС, достаточно смонтировать внутри этого дерева +/proc+,
+/sys+ и остальные API ФС, воспользоваться программой +chroot(1)+ для входа в
окружение, и выполнить соответствующий init-скрипт, запустив +/sbin/service+
внутри окружения.

Но в системах, использующих systemd, уже не~все так просто. Одно из важнейших
достоинств systemd состоит в том, что параметры среды, в которой запускаются
демоны, никак не~зависят от метода их запуска. В системах, использующих SysV
init, многие параметры среды выполнения (в частности, лимиты на системные
ресурсы, переменные окружения, и т.п.) наследуются от оболочки, из которой был
запущен init-скрипт. При использовании systemd ситуация меняется радикально:
пользователь просто уведомляет init-демона о необходимости запустить ту или иную
службу, и тот запускает демона в чистом, созданном <<с нуля>> и тщательно
настроенном окружении, параметры которого никак не~зависят от настроек среды, из
которой была отдана команда. Такой подход полностью отменяет традиционный метод
запуска демонов в chroot-окружениях: теперь демон порождается процессом init
(PID~1) и наследует корневой каталог от него, вне зависимости от того, находился
ли пользователь, отдавший команду на запуск, в chroot-окружении, или нет. Кроме
того, стоит особо отметить, что взаимодействие управляющих программ с systemd
происходит через сокеты, находящиеся в каталоге +/run/systemd+, так что
программы, запущенные в chroot-окружении, просто не~смогут взаимодействовать с
init-подсистемой (и это, в общем, неплохо, а если такое ограничение будет
создавать проблемы, его можно легко обойти, используя bind-монтирование).

В свете вышесказанного, возникает вопрос: как правильно использовать
chroot-окружения в системах на основе systemd? Что ж, постараемся дать подробный
и всесторонний ответ на этот вопрос.

Для начала, рассмотрим первое из перечисленных выше применений chroot: изоляция
в целях безопасности. Прежде всего, стоит заметить, что защита, предоставляемая
chroot'ом, весьма эфемерна и ненадежна, так как chroot не~является <<дорогой с
односторонним движением>>. Выйти из chroot-окружения сравнительно несложно, и
соответствующее предупреждение даже
\href{http://linux.die.net/man/2/chroot}{присутствует на странице руководства}.
Действительно эффективной защиты можно достичь, только сочетая chroot с другими
методиками. В большинстве случаев, это возможно только при наличии поддержки
chroot в самой программе. Прежде всего, корректное конфигурирование
chroot-окружения требует глубокого понимания принципов работы программы.
Например, нужно точно знать, какие каталоги нужно bind-монтировать из основной
системы, чтобы обеспечить все необходимые для работы программы каналы связи.  С
учетом вышесказанного, эффективная chroot-защита обеспечивается в том случае,
когда она реализована в коде самого демона. Именно разработчик лучше других
знает (\emph{обязан} знать), как правильно сконфигурировать chroot-окружение, и
какой минимальный набор файлов, каталогов и файловых систем необходим внутри
него для нормальной работы демона. Уже сейчас существуют демоны, имеющие
встроенную поддержку chroot.  К сожалению, в системе Fedora, установленной с
параметрами по умолчанию, таких демонов всего два:
\href{http://avahi.org/}{Avahi} и RealtimeKit. Оба они написаны одним очень
хитрым человеком ;-) (Вы можете собственноручно убедиться в этом, выполнив
команду +ls -l /proc/*/root+.)

Возвращаясь к теме нашего обсуждения: разумеется, systemd позволяет помещать
выбранных демонов в chroot, и управлять ими точно так же, как и другими.
Достаточно лишь указать параметр +RootDirectory=+ в соответствующем
service-файле. Например:
\begin{Verbatim}
[Unit]
Description=A chroot()ed Service

[Service]
RootDirectory=/srv/chroot/foobar
ExecStartPre=/usr/local/bin/setup-foobar-chroot.sh
ExecStart=/usr/bin/foobard
RootDirectoryStartOnly=yes
\end{Verbatim}

Рассмотрим этот пример подробнее. Параметр +RootDirectory=+ задает каталог, в
который производится chroot перед запуском исполняемого файла, заданного
параметром +ExecStart=+. Заметим, что путь к этому файлу должен быть указан
относительно каталога chroot (так что, в нашем случае, с точки зрения основной
системы, на исполнение будет запущен файл +/srv/chroot/foobar/usr/bin/foobard+).
Перед запуском демона будет вызван сценарий оболочки +setup-foobar-chroot.sh+,
который должен обеспечить подготовку chroot-окружения к запуску демона
(например, смонтировать в нем +/proc+ и/или другие файловые системы, необходимые
для работы демона). Указав +RootDirectoryStartOnly=yes+, мы задаем, что
+chroot()+ будет выполняться только перед выполнением файла из +ExecStart=+, а
команды из других директив, в частности, +ExecStartPre=+, будут иметь полный
доступ к иерархии файловых систем ОС (иначе наш скрипт просто не~сможет
выполнить bind-монтирование нужных каталогов). Более подробную информацию по
опциям конфигурации вы можете получить на
\href{http://0pointer.de/public/systemd-man/systemd.service.html}{страницах}
\href{http://0pointer.de/public/systemd-man/systemd.exec.html}{руководства}.

Поместив приведенный выше текст примера в файл
+/etc/systemd/system/foobar.service+, вы сможете запустить chroot'нутого демона
командой +systemctl start foobar.service+. Информацию о его текущем состоянии
можно получить с помощью команды +systemctl status foobar.service+. Команды
управления и мониторинга службы не~зависят от того, запущена ли она в chroot'е,
или нет. Этим systemd отличается от классического SysV init.

Новые ядра Linux поддерживают возможность создания независимых пространств имен
файловых систем (в дальнейшем FSNS, от <<file system namespaces>>). По
функциональности этот механизм аналогичен +chroot()+, однако предоставляет
гораздо более широкие возможности, и в нем отсутствуют проблемы с безопасностью,
характерные для chroot. systemd позволяет использовать при конфигурировании
юнитов некоторые возможности, предоставляемые FSNS. В частности, использование
FSNS часто является гораздо более простой и удобной альтернативой созданию
полновесных chroot-окружений. Используя директивы +ReadOnlyDirectories=+,
+InaccessibleDirectories=+, вы можете задать ограничения по использованию
иерархии файловых систем для заданной службы: ее корнем будет системный корневой
каталог, однако указанные в этих директивах подкаталоги будут доступны только 
для чтения или вообще недоступны для нее. Например:
\begin{Verbatim}
[Unit]
Description=A Service With No Access to /home

[Service]
ExecStart=/usr/bin/foobard
InaccessibleDirectories=/home
\end{Verbatim}

Такая служба будет иметь доступ ко всей иерархии файловых систем ОС, с
единственным исключением~--- она не~будет видеть каталог +/home+, что позволит
защитить данные пользователей от потенциальных хакеров. (Подробнее об этих
опциях можно почитать на
\href{http://0pointer.de/public/systemd-man/systemd.exec.html}{странице
руководства}.)

Фактически, FSNS по множеству параметров превосходят +chroot()+. Скорее всего,
Avahi и ReltimeKit в ближайшем будущем перейдут с +chroot()+ к использованию
FSNS.

Итак, мы рассмотрели вопросы использования chroot для обеспечения безопасности.
Переходим ко второму пункту: подготовка и управление образом операционной
системы при отладке, тестировании, компиляции, установке или восстановлении.

chroot-окружения, по сути, весьма примитивны: они изолируют только иерархии
файловых систем. Даже после chroot'а в определенный подкаталог, процесс
по-прежнему имеет полный доступ к системным вызовам, может убивать процессы,
запущенные в основной системе, и т.п. Вследствие этого, запуск полноценной ОС
(или ее части) внутри chroot'а несет угрозу для хост-системы: у гостя и хоста
отличается лишь содержимое файловой системы, все остальное у них общее.
Например, если вы обновляете дистрибутив, установленный в chroot-окружении, и
пост-установочный скрипт пакета отправляет +SIGTERM+ процессу init для его
перезапуска\footnote{Прим. перев.: во избежание путаницы отметим, что перезапуск
процесса init (PID~1) <<на лету>> при получении +SIGTERM+ поддерживается только
в systemd, в классическом SysV init такой возможности нет.}, на него среагирует
именно хост-система! Кроме того, хост и chroot'нутая система будут иметь общую
разделяемую память SysV (SysV shared memory), общие сокеты из абстрактных
пространств имен (abstract namespace sockets) и другие элементы IPC. Для
отладки, тестирования, компиляции, установки и восстановлении ОС не~требуется
абсолютно неуязвимая изоляция~--- нужна лишь надежная защита от
\emph{случайного} воздействия на ОС хоста изнутри chroot-окружения, иначе вы
можете получить целый букет проблем, как минимум, от пост-инсталляционных
скриптов при установке пакетов в chroot-окружении.

systemd имеет целый ряд возможностей, полезных для работы с chroot-системами:

Прежде всего, управляющая программа +systemctl+ автоматически определяет, что
она запущена в chroot-системе. В такой ситуации будут работать только команды
+systemctl enable+ и +systemctl disable+, во всех остальных случаях +systemctl+
просто не~будет ничего делать, возвращая код успешного завершения операции.
Таким образом, пакетные скрипты смогут включить/отключить запуск <<своих>> служб
при загрузке (или в других ситуациях), однако команды наподобие
+systemctl restart+ (обычно выполняется при обновлении пакета) не~дадут никакого
эффекта внутри chroot-окружения\footnote{Прим. перев.: автор забывает отметить
не~вполне очевидный момент: такое поведение +systemctl+ проявляется только в
<<мертвых>> окружениях, т.е. в тех, где не~запущен процесс init, и
соответственно отсутствуют управляющие сокеты в +/run/systemd+. Такая ситуация
возникает, например, при установке системы в chroot через
debootstrap/febootstrap. В этом случае возможности +systemctl+ ограничиваются
операциями с символьными ссылками, определяющими триггеры активации юнитов, т.е.
выполнением действий +enable+ и +disable+, не~требующих непосредственного
взаимодействия с процессом init.}.

Однако, куда более интересные возможности предоставляет программа
\href{http://0pointer.de/public/systemd-man/systemd-nspawn.html}{systemd-nspawn},
входящая в стандартный комплект поставки systemd. По сути, это улучшенный аналог
+chroot(1)+~--- она не~только подменяет корневой каталог, но и создает отдельные
пространства имен для дерева файловых систем (FSNS) и для идентификаторов
процессов (PID NS), предоставляя легковесную реализацию системного
контейнера\footnote{Прим. перев.: используемые в +systemd-nspawn+ механизмы 
ядра Linux, такие, как FS NS и PID NS, также лежат в основе
\href{http://lxc.sourceforge.net/}{LXC}, системы контейнерной изоляции для
Linux, которая позиционируется как современная альтернатива классическому
\href{http://wiki.openvz.org/Main_Page}{OpenVZ}. Стоит отметить, что LXC 
ориентирована прежде всего на создание независимых виртуальных окружений,
с поддержкой раздельных сетевых стеков, ограничением на ресурсы, сохранением
настроек и т.п., в то время как +systemd-nspawn+ является лишь более удобной и
эффективной заменой команды +chroot(1)+, предназначенной прежде всего для
развертывания, восстановления, сборки и тестирования операционных систем. Далее
автор разъясняет свою точку зрения на этот вопрос.}.
+systemd-nspawn+ проста в использовании как +chroot(1)+, однако изоляция
от хост-системы является более полной и безопасной. Всего одной командой
вы можете загрузить внутри контейнера \emph{полноценную} ОС (на базе systemd
или SysV init). Благодаря использованию независимых пространств идентификаторов
процессов, процесс init внутри контейнера получит PID~1, что позволит работать
ему в штатном режиме. Также, в отличие от +chroot(1)+, внутри окружения 
будут автоматически смонтированы +/proc+ и +/sys+.

Следующий пример иллюстрирует возможность запустить Debian в на Fedora-хосте
всего тремя командами:
\begin{Verbatim}
# yum install debootstrap
# debootstrap --arch=amd64 unstable debian-tree/
# systemd-nspawn -D debian-tree/
\end{Verbatim}

Вторая из этих команд обеспечивает развертывание в подкаталоге +./debian-tree/+
файловой структуры дистрибутива Debian, после чего третья команда запускает
внутри полученной системы процесс командной оболочки. Если вы хотите запустить
внутри контейнера полноценную ОС, воспользуйтесь командой
\begin{Verbatim}
# systemd-nspawn -D debian-tree/ /sbin/init
\end{Verbatim}

После быстрой загрузки вы получите приглашение оболочки, запущенной внутри
полноценной ОС, функционирующей в контейнере. Изнутри контейнера невозможно
увидеть процессы, которые находятся вне его. Контейнер сможет пользоваться сетью
хоста, однако не~имеет возможности изменить ее настройки (это может привести к
серии ошибок в процессе загрузки гостевой ОС, но ни~одна из этих ошибок
не~должна быть критической). Контейнер получает доступ к +/sys+ и +/proc/sys+,
однако, во избежание вмешательства контейнера в конфигурацию ядра и аппаратного
обеспечения хоста, эти каталоги будут смонтированы только для чтения. Обратите
внимание, что эта защита блокирует лишь \emph{случайные}, \emph{непредвиденные}
попытки изменения параметров. При необходимости, процесс внутри контейнера,
обладающий достаточными полномочиями, сможет перемонтировать эти файловые
системы в режиме чтения-записи.

Итак, что же такого хорошего в +systemd-nspawn+?
\begin{enumerate}
	\item Использовать эту утилиту очень просто. Вам даже не~нужно вручную монтировать
		внутри окружения +/proc+ и +/sys+~--- она сделает это за вас, а
		ядро автоматически отмонтирует их, когда последний процесс
		контейнера завершится.
	\item Обеспечивается надежная изоляция, предотвращающая случайные
		изменения параметров ОС хоста изнутри контейнера.
	\item Теперь вы можете загрузить внутри контейнера полноценную ОС, а
		не~одну-единственную оболочку.
	\item Эта утилита очень компактна и присутствует везде, где установлен
		systemd. Она не~требует специальной установки и настройки.
\end{enumerate}

systemd уже подготовлен для работы внутри таких контейнеров. Например, когда
подается команда на выключение системы внутри контейнера, systemd на последнем
шаге вызывает не~+reboot()+, а просто +exit()+.

Стоит отметить, что +systemd-nspawn+ все же не~является полноценной системой
контейнерной виртуализации/изоляции~--- если вам нужно такое решение,
воспользуйтесь \href{ http://lxc.sourceforge.net/}{LXC}. Этот проект использует
те же самые механизмы ядра, но предоставляет куда более широкие возможности,
включая виртуализацию сети. Могу предложить такую аналогию: +systemd-nspawn+ как
реализация контейнера похожа на GNOME~3~--- компактна и проста в использовании,
опций для настройки очень мало. В то время как LXC больше похож на KDE: опций
для настройки больше, чем строк кода. Я создал +systemd-nspawn+ специально для
тестирования, отладки, сборки, восстановления. Именно для этих задач вам стоит
ее использовать~--- она неплохо с ними справляется, куда лучше, чем +chroot(1)+.

Что ж, пора заканчивать. Итак:
\begin{enumerate}
	\item Использование +chroot()+ для обеспечения безопасности дает
		наилучший результат, когда оно реализовано непосредственно в
		коде самой программы.
	\item +ReadOnlyDirectories=+ и +InaccessibleDirectories=+ могут быть
		удобной альтернативой созданию полновесных chroot-окружений.
	\item Если вам нужно поместить в chroot-окружение какую-либо службу,
		воспользуйтесь опцией +RootDirectory=+.
	\item +systemd-nspawn+~--- очень неплохая штука.
	\item chroot'ы убоги, FSNS~---
		\href{http://ru.wikipedia.org/wiki/Leet}{1337}.
\end{enumerate}

И все это уже сейчас доступно в Fedora~15.

\section{Поиск виновных}

Fedora~15\footnote{Величайший в истории релиз свободной ОС}
является первым релизом Fedora, использующим systemd в качестве системы
инициализации по умолчанию. Основной нашей целью при работе над выпуском F15
является обеспечение полной взаимной интеграции и корректной работы всех
компонентов. При подготовке следующего релиза, F16, мы сконцентрируемся на
дальнейшей полировке и ускорении системы. Для этого мы подготовили ряд
инструментов (доступных уже в F15), которые должны помочь нам в поиске проблем,
связанных с процессом загрузки. В этой статье я попытаюсь рассказать о том, как
найти виновников медленной загрузки вашей системы, и о том, что с ними делать
дальше.

Первый инструмент, который мы можем вам предложить, очень прост: по завершении
загрузки, systemd регистрирует в системном журнале информацию о суммарном
времени загрузки:
\begin{Verbatim}
systemd[1]: Startup finished in 2s 65ms 924us (kernel) + 2s 828ms 195us (initrd)
+ 11s 900ms 471us (userspace) = 16s 794ms 590us.
\end{Verbatim}

Эта запись означает следующее: на инициализацию ядра (до момента запуска initrd,
т.е. dracut) ушло 2 секунды. Далее, чуть менее трех секунд работал initrd. И
наконец, почти 12 секунд было потрачено systemd на запуск программ из
пространства пользователя. Итоговое время, начиная с того момента, как загрузчик
передал управление коду ядра, до того момента, как systemd завершил все
операции, связанные с загрузкой системы, составило почти 17 секунд.  Казалось
бы, смысл этого числа вполне очевиден\ldots{} Однако не~стоит делать поспешных
выводов. Прежде всего, сюда не~входит время, затраченное на
инициализацию вашего сеанса в GNOME, так как эта задача уже выходит за рамки
задач процесса init. Кроме того, в этом показателе учитывается только время
работы systemd, хотя часто бывает так, что некоторые демоны продолжают
\emph{свою} работу по инициализации уже после того, как секундомер остановлен.
Проще говоря: приведенные числа позволяют лишь оценить общую скорость
загрузки, однако они не~являются точной характеристикой длительности процесса.

Кроме того, эта информация носит поверхностный характер: она не~сообщает, какие
именно системные компоненты заставляют systemd ждать так долго. Чтобы исправить 
это упущение, мы ввели команду +systemd-analyze blame+:
\begin{Verbatim}
$ systemd-analyze blame
  6207ms udev-settle.service
  5228ms cryptsetup@luks\x2d9899b85d\x2df790\x2d4d2a\x2da650\x2d8b7d2fb92cc3.service
   735ms NetworkManager.service
   642ms avahi-daemon.service
   600ms abrtd.service
   517ms rtkit-daemon.service
   478ms fedora-storage-init.service
   396ms dbus.service
   390ms rpcidmapd.service
   346ms systemd-tmpfiles-setup.service
   322ms fedora-sysinit-unhack.service
   316ms cups.service
   310ms console-kit-log-system-start.service
   309ms libvirtd.service
   303ms rpcbind.service
   298ms ksmtuned.service
   288ms lvm2-monitor.service
   281ms rpcgssd.service
   277ms sshd.service
   276ms livesys.service
   267ms iscsid.service
   236ms mdmonitor.service
   234ms nfslock.service
   223ms ksm.service
   218ms mcelog.service
...
\end{Verbatim}

Она выводит список юнитов systemd, активированных при загрузке, с указанием
времени инициализации для каждого из них. Список отсортирован по убыванию этого
времени, поэтому наибольший интерес для нас представляют первые строчки. В нашем
случае это +udev-settle.service+ и
+cryptsetup@luks\x2d9899b85d\x2df790\x2d4d2a\x2da650\x2d8b7d2fb92cc3.service+,
инициализация которых занимает более одной секунды. Стоит отметить, что к
анализу вывода команды +systemd-analyze blame+ тоже следует подходить с
осторожностью: она не~поясняет, \emph{почему} тот или иной юнит тратит
столько-то времени, она лишь констатирует факт, что время было затрачено. Кроме
того, не~стоит забывать, что юниты могут запускаться параллельно. В частности,
если две службы были запущены одновременно, то время их инициализации будет
значительно меньше, чем сумма времен инициализации каждой из них.

Рассмотрим повнимательнее первого осквернителя нашей загрузки: службу
+udev-settle.service+. Почему ей требуется так много времени для запуска, и что
мы можем с этим сделать? Эта служба выполняет очень простую задачу: она ожидает,
пока udev завершит опрос устройств, после чего завершается. Опрос же устройств
может занимать довольно много времени. Например, в нашем случае опрос устройств
занимает более 6~секунд из-за подключенного к компьютеру 3G-модема, в котором
отсутствует SIM-карта. Этот модем очень долго отвечает на запросы udev. Опрос
устройств является частью схемы, обеспечивающей работу ModemManager'а и
позволяющей NetworkManager'у упростить для вас настройку 3G. Казалось бы,
очевидно, что виновником задержки является именно ModemManager, так как опрос
устройств для него занимает слишком много времени. Но такое обвинение будет
заведомо ошибочным. Дело в том, что опрос устройств очень часто оказывается довольно
длительной процедурой. Медленный опрос 3G-устройств для ModemManager является
частным случаем, отражающим это общее правило. Хорошая система опроса устройств
обязательно должна учитывать тот факт, что операция опроса любого из устройств
может затянуться надолго. Истинной причиной наших проблем является необходимость
ожидать завершения опроса устройств, т.е., наличие службы
+udev-settle.service+ как обязательной части нашего процесса загрузки.

Но почему эта служба вообще присутствует в нашем процессе загрузки? На самом
деле, мы можем прекрасно обойтись и без нее. Она нужна лишь как часть
используемой в Fedora схемы инициализации устройств хранения, а именно,
набора скриптов, выполняющих настройку LVM, RAID и multipath-устройств. На
сегодняшний день, реализация этих систем не~поддерживает собственного механизма
поиска и опроса устройств, и поэтому их запуск должен производиться только после
того, как эта работа уже проделана~--- тогда они могут просто последовательно
перебирать все диски. Однако, такой подход противоречит одному из
фундаментальных требований к современным компьютерам:
возможности подключать и отключать оборудование в любой момент, как при
выключенном компьютере, так и при включенном. Для некоторых
технологий невозможно точно определить момент завершения формирования списка устройств
(например, это характерно для USB и iSCSI), и поэтому процесс опроса
таких устройств обязательно должен включать некоторую фиксированную задержку,
гарантирующую, что все подключенные устройства успеют отозваться. С точки зрения
скорости загрузки это, безусловно, негативный эффект: соответствующие скрипты
заставляют нас ожидать завершения опроса устройств, хотя большинство этих
устройств не~являются необходимыми на данной стадии загрузки. В частности, в
рассматриваемой нами системе LVM, RAID и multipath вообще
не~используются!\footnote{Наиболее правильным решением в данном случае будет
ожидание событий подключения устройств (реализованное через libudev или
аналогичную технологию) с соответствующей обработкой каждого такого события~---
подобный подход позволит нам продолжить загрузку, как только будут готовы все
устройства, необходимые для ее продолжения. Для того, чтобы загрузка была
быстрой, мы должны ожидать завершения инициализации только тех устройств, которые
\emph{действительно} необходимы для ее продолжения на данной стадии. Ожидать
остальные устройства в данном случае смысла нет. Кроме того, стоит отметить, что
в числе программ, не~приспособленных для работы с динамически меняющимся
оборудованием и построенных в предположении о неизменности списка устройств,
кроме служб хранения, есть и другие подсистемы. Например, в нашем случае работа 
initrd занимает так много времени главным образом из-за того, что для запуска
Plymouth необходимо дождаться завершения инициализации всех устройств
видеовывода. По неизвестной причине (во всяком случае, неизвестной для меня)
подгрузка модулей для моих видеокарт Intel занимает довольно продолжительное
время, что приводит к беспричинной задержке процесса загрузки. (Нет, я протестую
не~против опроса устройств, но против необходимости ожидать его завершения, чтобы
продолжить загрузку.)}

С учетом вышесказанного, мы можем спокойно исключить +udev-settle.service+ из
нашего процесса загрузки, так как мы не~используем ни~LVM, ни~RAID,
ни~multipath. Замаскируем эти службы, чтобы увеличить скорость загрузки:
\begin{Verbatim}
# ln -s /dev/null /etc/systemd/system/udev-settle.service
# ln -s /dev/null /etc/systemd/system/fedora-wait-storage.service
# ln -s /dev/null /etc/systemd/system/fedora-storage-init.service
# systemctl daemon-reload
\end{Verbatim}

После перезагрузки мы видим, что загрузка стала на одну секунду быстрее. Но
почему выигрыш оказался таким маленьким? Из-за второго осквернителя нашей
загрузки~--- cryptsetup. На рассматриваемой нами системе зашифрован раздел
+/home+. Специально для тестирования я записал пароль в файл, чтобы исключить
влияние скорости ручного набора пароля. К сожалению, для подключения
шифрованного раздела cryptsetup требует более пяти секунд. Будем ленивы, и
вместо того, чтобы исправлять ошибку в cryptsetup\footnote{На самом деле, я
действительно пытался исправить эту ошибку. Задержки в cryptsetup, по
моим наблюдениям, обусловлены, главным образом, слишком большим значением по
умолчанию для опции +--iter-time+. Я попытался доказать сопровождающим пакета
cryptsetup, что снижение этого времени с 1 секунды до 100~миллисекунд
не~приведет к трагическим последствиям для безопасности, однако они мне
не~поверили.}, попробуем найти обходной путь\footnote{Вообще-то я предпочитаю
решать проблемы, а не~искать пути для их обхода, однако в данном конкретном
случае возникает отличная возможность продемонстрировать одну интересную
возможность systemd\ldots{}}. Во время загрузки systemd должен дождаться, пока
все файловые системы, перечисленные в +/etc/fstab+ (кроме помеченных как
+noauto+) будут обнаружены, проверены и смонтированы. Только после этого systemd
может продолжить загрузку и приступить к запуску служб. Но первое обращение
к +/home+ (в отличие, например, от +/var+), происходит на поздней стадии
процесса загрузки (когда пользователь входит в систему). Следовательно, нам
нужно сделать так, чтобы этот каталога автоматически монтировался при загрузке,
но процесс загрузки не~ожидал завершения работы +cryptsetup+, +fsck+ и +mount+
для этого раздела. Как же сделать точку монтирования доступной, не~ожидая, пока
завершится процесс монтирования? Этого можно достичь, воспользовавшись
магической силой systemd~--- просто добавим опцию монтирования
+comment=systemd.automount+ в +/etc/fstab+. После этого, systemd будет создавать
в +/home+ точку автоматического монтирования, и при первом же обращении к этому
каталогу, если файловая система еще не~будет готова к работе, systemd подготовит
соответствующее устройство, проверит и смонтирует ее.

После внесения изменений в +/etc/fstab+ и перезагрузки мы получаем:
\begin{Verbatim}
systemd[1]: Startup finished in 2s 47ms 112us (kernel) + 2s 663ms 942us (initrd)
+ 5s 540ms 522us (userspace) = 10s 251ms 576us.
\end{Verbatim}

Прекрасно! Несколькими простыми действиями мы выиграли почти семь секунд. И эти
два изменения исправляют только две наиболее очевидные проблемы. При более
аккуратном и детальном исследовании, обнаружится еще множество моментов, которые
можно улучшить. Например, на другом моем компьютере, лаптопе X300 двухлетней
давности (и даже два года назад он был не~самым быстрым на Земле), после
небольшой доработки, время загрузки до полноценной среды GNOME
составило около четырех секунд, и это еще не~предел совершенства.

+systemd-analyze blame+~--- простой и удобный инструмент для поиска медленно
запускающихся служб, однако он обладает одним существенным недостатком: он
не~показывает, насколько эффективно параллельный запуск снижает потерю времени
для медленно запускающихся служб. Чтобы вы могли наглядно оценить этот фактор,
мы подготовили команду +systemd-analyze plot+. Использовать ее очень просто:
\begin{Verbatim}
$ systemd-analyze plot > plot.svg
$ eog plot.svg
\end{Verbatim}

Она создает наглядные диаграммы, показывающие моменты запуска служб и время,
затраченное на их запуск, по отношению к другим службам. На текущий момент, она
не~показывает явно, кто кого ожидает, но догадаться обычно несложно.

Чтобы продемонстрировать эффект, порожденный двумя нашими оптимизациями,
приведем ссылки на соответствующие графики:
\href{http://0pointer.de/public/blame.svg}{до} и
\href{http://0pointer.de/public/blame2.svg}{после} (для полноты описания приведем также и
соответствующие данные +systemd-analyze blame+:
\href{http://0pointer.de/public/blame.txt}{до} и
\href{http://0pointer.de/public/blame2.txt}{после}).

У наиболее эрудированных читателей может возникнуть вопрос: как это соотносится с
программой \href{https://github.com/mmeeks/bootchart}{bootchart}? Действительно,
+systemd-analyze plot+ и +bootchart+ рисуют похожие графики. Однако, bootchart
является намного более мощным инструментом~--- он детально показывает, что
именно происходило во время загрузки, и отображает соответствующие графики
использования процессора и ввода-вывода. +systemd-analyze plot+ оперирует более
высокоуровневой информацией: сколько времени затратила та или иная служба во
время запуска, и какие службы были вынуждены ее ожидать. Используя оба эти
инструмента, вы значительно упростите себе поиск причин замедления вашей
загрузки.

Но прежде, чем вы вооружитесь описанными здесь средствами и начнете
отправлять багрепорты авторам и сопровождающим программ, которые замедляют вашу
загрузку, обдумайте все еще раз. Эти инструменты предоставляют вам <<сырую>>
информацию. Постарайтесь не~ошибиться, интерпретируя ее. Например, в
при рассмотрении приведенного выше примера я показал, что проблема была вовсе
не~в +udev-settle.service+, и не~в опросе устройств для ModemManager'а, а в 
неудачной реализации подсистемы хранения, требующей ожидать окончания опроса
устройств для продолжения загрузки. Именно эту проблему и нужно исправлять.
Поэтому постарайтесь правильно определить источник проблем. Возлагайте вину на
тех, кто действительно виноват.

Как уже говорилось выше, все три описанных здесь инструмента доступны в
Fedora~15 <<из коробки>>.

Итак, какие же выводы мы можем сделать из этой истории?
\begin{itemize}
	\item +systemd-analyze+~--- отличный инструмент. Фактически, это
		встроенный профилировщик systemd.
	\item Постарайтесь не~ошибиться, интерпретируя вывод профилировщика!
	\item Всего два небольших изменения могут ускорить загрузку системы на
		семь секунд.
	\item Программное обеспечение, не~способное работать с динамически
		меняющимся набором устройств, создает проблемы, и должно быть
		исправлено.
	\item Принудительное использование в стандартной установке Fedora~15
		промышленной реализации подсистем хранения, возможно, является
		не~самым правильным решением.
\end{itemize}

\section{Новые конфигурационные файлы}

Одно из ключевых достоинств
\href{http://www.freedesktop.org/wiki/Software/systemd}{systemd}~--- наличие
полного набора программ, необходимых на ранних стадиях загрузки, причем эти
программы написаны на простом, быстром, надежном и легко поддающемся
распараллеливанию языке C. Теперь можно отказаться от <<простыней>>
shell-скриптов, разработанных для этих задач различными дистрибутивами. Наш
<<Проект нулевой оболочки>>\footnote{Наш девиз~--- <<Первой оболочкой,
запускающейся при старте системы, должна быть GNOME shell>>. Формулировка
оставляет желать лучшего, но все же неплохо передает основную идею.} увенчался
полным успехом. Уже сейчас возможности предоставляемого нами инструментария
покрывают практически все нужды настольных и встраиваемых систем, а также
б\'{о}льшую часть потребностей серверов:
\begin{itemize}
	\item Проверка и монтирование всех файловых систем.
	\item Обновление и активация квот на всех файловых системах.
	\item Установка имени хоста.
	\item Настройка сетевого интерфейса обратной петли (+lo+).
	\item Подгрузка правил SELinux, обновление
		меток безопасности в динамических каталогах +/run+ и +/dev+.
	\item Регистрация в ядре дополнительных бинарных форматов (например,
		Java, Mono, WINE) через API-файловую систему +binfmt_misc+.
	\item Установка системной локали.
	\item Настройка шрифта и раскладки клавиатуры в консоли.
	\item Создание, очистка, удаление временных файлов и каталогов.
	\item Применение предписанных в +/etc/fstab+ опций к смонтированным
		ранее API-файловым системам.
	\item Применение настроек +sysctl+.
	\item Поддержка технологии упреждающего чтения (read ahead), включая
		автоматический сбор информации.
	\item Обновление записей в +utmp+ при включении и выключении системы.
	\item Сохранение и восстановление затравки для генерации случайных чисел
		(random seed).
	\item Принудительная загрузка указанных модулей ядра.
	\item Поддержка шифрованных дисков и разделов.
	\item Автоматический запуск +getty+ на serial-консолях.
	\item Взаимодействие с Plymouth.
	\item Создание уникального идентификатора системы.
	\item Установка часового пояса.
\end{itemize}

В стандартной установке Fedora~15 запуск shell-скриптов требуется только для
некоторых устаревших служб, а также для подсистемы хранения данных (поддержка
LVM, RAID и multipath). Если они вам не~нужны, вы легко можете отключить их, и
наслаждаться загрузкой, полностью очищенной от shell-костылей (я это сделал уже
давно). Такая загрузка является уникальной возможностью Linux-систем.

Большинство перечисленных выше компонентов настраиваются через конфигурационные
файлы в каталоге +/etc+. Некоторые из этих файлов стандартизированы для всех
дистрибутивов, и поэтому реализация их поддержки в наших инструментах
не~представляла особого труда. Например, это относится к файлам +/etc/fstab+,
+/etc/crypttab+, +/etc/sysctl.conf+. Однако множество других, нестандартно
расположенных файлов и каталогов вынуждали нас добавлять в код огромное
количество операторов +#ifdef+, чтобы обеспечить поддержку различных вариантов
расположения конфигураций в разных дистрибутивах. Такой положение дел сильно
усложняет жизнь нам всем, и при этом ничем не~оправдано~--- все эти файлы решают
одни и те же задачи, но делают это немного по-разному.

Чтобы улучшить ситуацию и установить единый стандарт расположения базовых
конфигурационных файлов во всех дистрибутивах, мы заставили systemd пользоваться
дистрибутивно-специфическими конфигурациями только в качестве \emph{резервного}
варианта~--- основным источником информации становится определенный нами
стандартный набор конфигурационных файлов. Разумеется, там, где это возможно, мы
старались не~придумывать чего-то принципиально нового, а брали лучшее из
решений, предложенных существующими дистрибутивами. Ниже приводится небольшой
обзор этого нового набора конфигурационных файлов, поддерживаемых systemd во
всех дистрибутивах:
\begin{itemize}
	\item
		\hreftt{http://0pointer.de/public/systemd-man/hostname.html}{/etc/hostname}:
		имя хоста для данной системы. Одна из наиболее простых и важных
		системных настроек. В разных дистрибутивах оно настраивалось
		по-разному: Fedora использовала +/etc/sysconfig/network+,
		OpenSUSE~--- +/etc/HOSTNAME+, Debian~--- +/etc/hostname+. Мы
		остановились на варианте, предложенном Debian.
	\item
		\hreftt{http://0pointer.de/public/systemd-man/vconsole.conf.html}{/etc/vconsole.conf}:
		конфигурация раскладки клавиатуры и шрифта для консоли.
	\item
		\hreftt{http://0pointer.de/public/systemd-man/locale.conf.html}{/etc/locale.conf}:
		конфигурация общесистемной локали.
	\item
		\hreftt{http://0pointer.de/public/systemd-man/modules-load.d.html}{/etc/modules-load.d/*.conf}:
		каталог\footnote{Прим. перев.: для описания этого и трех
		последующих каталогов автор пользуется термином <<drop-in
		directory>>. Этот термин означает каталог, в который можно
		поместить множество независимых файлов настроек, и при чтении
		конфигурации все эти файлы будут обработаны (впрочем, часто
		накладывается ограничение~--- обрабатываются только файлы с
		именами, соответствующими маске, обычно +*.conf+). Такой подход
		позволяет значительно упростить процесс как ручного, так и
		автоматического конфигурирования различных компонентов~--- для
		внесения изменений в настройки уже не~нужно редактировать
		основной конфигурационный файл, достаточно лишь
		скопировать/переместить в нужный каталог небольшой файл с
		указанием специфичных параметров.} для перечисления модулей
		ядра, которые нужно принудительно подгрузить при загрузке
		(впрочем, необходимость в этом возникает достаточно редко).
	\item
		\hreftt{http://0pointer.de/public/systemd-man/sysctl.d.html}{/etc/sysctl.d/*.conf}:
		каталог для задания параметров ядра (+sysctl+). Дополняет
		классический конфигурационный файл +/etc/sysctl.conf+.
	\item
		\hreftt{http://0pointer.de/public/systemd-man/tmpfiles.d.html}{/etc/tmpfiles.d/*.conf}:
		каталог для управления настройками временных файлов (systemd
		обеспечивает создание, очистку и удаление временных файлов и
		каталогов, как во время загрузки, так и во время работы
		системы).
	\item
		\hreftt{http://0pointer.de/public/systemd-man/binfmt.d.html}{/etc/binfmt.d/*.conf}:
		каталог для регистрации дополнительных бинарных форматов
		(например, форматов Java, Mono, WINE).
	\item
		\hreftt{http://0pointer.de/public/systemd-man/os-release.html}{/etc/os-release}:
		стандарт для файла, обеспечивающего идентификацию дистрибутива и
		его версии. Сейчас различные дистрибутивы используют для этого
		разные файлы (например, +/etc/fedora-release+ в Fedora), и
		поэтому для решения такой простой задачи, как вывод имени
		дистрибутива, необходимо использовать базу данных, содержащую
		перечень возможных названий файлов. Проект LSB попытался создать
		такой инструмент~---
		\hreftt{http://refspecs.freestandards.org/LSB_3.1.0/LSB-Core-generic/LSB-Core-generic/lsbrelease.html}{lsb\_release}~---
		однако реализация столь простой функции через скрипт на Python'е
		является не~самым оптимальным решением. Чтобы исправить
		сложившуюся ситуацию, мы решили перейти к единому простому
		формату представления этой информации.
	\item
		\hreftt{http://0pointer.de/public/systemd-man/machine-id.html}{/etc/machine-id}:
		файл с идентификатором данного компьютера (перекрывает
		аналогичный идентификатор D-Bus). Гарантируется, что в любой
		системе, использующей systemd, этот файл будет существовать и
		содержать корректную информацию (если его нет, он автоматически
		создается при загрузке). Мы вынесли этот файл из-под эгиды
		D-Bus, чтобы упростить решение множества задач, требующих
		наличия уникального и постоянного идентификатора компьютера.
	\item
		\hreftt{http://0pointer.de/public/systemd-man/machine-info.html}{/etc/machine-info}:
		новый конфигурационный файл, хранящий информации о полном имени
		(описательном) хоста (например, <<Компьютер Леннарта>>) и
		значке, которым он будет обозначаться в графических оболочках,
		работающих с сетью (раньше этот значок мог определяться,
		например, файлом +/etc/favicon.png+). Данный конфигурационный
		файл обслуживается демоном
		\hreftt{http://www.freedesktop.org/wiki/Software/systemd/hostnamed}{systemd-hostnamed}.
\end{itemize}

Одна из важнейших для нас задач~--- убедить \emph{вас} использовать эти новые
конфигурационные файлы в ваших инструментах для настройки системы. Если ваши
конфигурационные фронтенды будут использовать новые файлы, а не~их старые
аналоги, это значительно облегчит портирование таких фронтендов между
дистрибутивами, и вы внесете свой вклад в стандартизацию Linux. В конечном счете
это упростит жизнь и администраторам, и пользователям. Разумеется, на текущий
момент эти файлы полностью поддерживаются только дистрибутивами, основанными на
systemd, но уже сейчас в их число входят практически все ключевые дистрибутивы,
\href{http://www.ubuntu.com/}{за исключением
одного}\footnote{Прим. перев.: в конце 2010~года энтузиаст Andrew Edmunds
\href{http://cgit.freedesktop.org/systemd/commit/?id=858dae181bb5461201ac1c04732d3ef4c67a0256}{добавил}
в systemd базовую поддержку Ubuntu и
\href{https://wiki.ubuntu.com/systemd}{подготовил} соответствующие пакеты,
однако его инициатива не~встретила поддержки среди менеджеров Canonical. На
момент написания этих строк (май 2011 года) проект остается заброшенным уже пять
месяцев (с середины декабря 2010~г.).}. В этом есть что-то от <<пролемы курицы и
яйца>>: стандарт становится начинает работать как стандарт только тогда, когда
ему начинают следовать. В будущем мы намерены аккуратно форсировать процесс
перехода на новые конфигурационные файлы: поддержка старых файлов будет удалена
из systemd. Разумеется, этот процесс будет идти медленно, шаг за шагом. Но
конечной его целью является переход всех дистрибутивов на единый набор базовых
конфигурационных файлов.

Многие из этих файлов используются не~только программами для настройки системы,
но и апстримными проектами. Например, мы предлагаем проектам Mono, Java, WINE и
другим помещать конфигурацию для регистрации своих бинарных форматов в
+/etc/binfmt.d/+ средствами их собственной сборочной системы. Специфичные для
дистрибутивов механизмы поддержки бинарных форматов больше не~нужны, и ваш
проект будет работать одинаково хорошо во всех дистрибутивах. Аналогичное
предложение мы обращаем и ко всем разработчикам программ, которым требуется
автоматическое создание/очистка временных файлов и каталогов,
например, в каталоге +/run+ (\href{http://lwn.net/Articles/436012/}{ранее
известном} как +/var/run+). Таким проектам достаточно просто поместить
соответствующий конфигурационный файл в +/etc/tmpfiles.d/+, тоже средствами
собственной сборочной системы. Помимо прочего, подобный подход позволит
увеличить скорость загрузки, так как, в отличие от SysV, не~требует множества
shell-скриптов, выполняющих тривиальные задачи (регистрация бинарных форматов,
удаление/создание временных файлов/каталогов и т.п.). И пример того случая,
когда апстримная поддержка стандартной конфигурации дала бы огромные
преимущества~--- X11 (и его аналоги) могли бы устанавливать раскладку клавиатуры
на основании данных из +/etc/vconsole.conf+.

Разумеется, я понимаю, что отнюдь не~всех полностью устроят выбранные нами имена
и форматы конфигурационных файлов. Но нам все же нужно было что-то выбрать, и мы
выбрали то, что должно устроить большинство людей. Форматы конфигурационных
файлов максимально просты, и их можно легко читать и записывать даже из
shell-скриптов. Да, +/etc/bikeshed.conf+ могло бы быть неплохим именем
для файла конфигурации!\footnote{Прим. перев.: здесь автор намекает на
\href{http://en.wikipedia.org/wiki/Parkinson's_Law_of_Triviality}{Паркинсоновский
Закон Тривиальности}, который гласит, что самые жаркие споры возникают вокруг
наиболее простых вопросов. В частности, в качестве примера Паркинсон приводит
обсуждение строительства атомной электростанции и гаража для велосипедов (bike
shed)~--- если первое из этих решений принимается довольно быстро, то вокруг
второго разгорается множество дискуссий по самым разным аспектам.}

\textbf{Помогите нам стандартизировать Linux! Используйте новые конфигурационные
файлы! Поддерживайте их в апстриме, поддерживайте их во всех дистрибутивах!}

Да, и если у вас возникнет такой вопрос: да, все эти файлы так или иначе
обсуждались с разными разработчиками из различных дистрибутивов. И некоторые из
этих разработчиков планируют обеспечить поддержку новой конфигурации даже в
системах без systemd.

\section{О судьбе /etc/sysconfig и /etc/default}

В дистрибутивах, основанных на Red Hat и SUSE, это каталог называется
+/etc/sysconfig+. В дистрибутивах на базе Debian, его зовут +/etc/default+.
Во многих других дистрибутивах также присутствуют каталоги похожего назначения.
Связанные с ними вопросы неоднократно появляются в дискуссиях пользователей и
разработчиков systemd. В этой статье мне хотелось бы рассказать, что я, как
разработчик systemd, думаю об этих каталогах, и пояснить, почему, на мой взгляд,
от них лучше отказаться. Стоит отметить, что это мое личное мнение, и оно
может не~совпадать с позицией проекта Fedora или моего работодателя.

Начнем с небольшого исторического экскурса. Каталог +/etc/sysconfig+ появился в
дистрибутивах Red Hat и SUSE задолго до того, как я присоединился к этим
проектам~--- иными словами, это было очень давно. 
Некоторое время спустя, в Debian появился аналогичный по смыслу каталог
+/etc/default+. Многие дистрибутивы используют такие каталоги, называя их
по-разному. Они имеются даже в некоторых ОС семейства Unix. (Например, в SCO.
Если эта тема вас заинтересовала~--- рекомендую обратиться к вашему знакомому
ветерану Unix, он расскажет гораздо подробнее и интереснее, чем я.) Несмотря на
то, что подобные каталоги широко используются в Linux и Unix, они совершенно
не~стандартизированы~--- ни в POSIX, ни в LSB/FHS, и результате мы имеем целый
зоопарк их различных реализаций в разных дистрибутивах.

Назначение этих каталогов определено весьма расплывчато. Абсолютное большинство
находящихся в них файлов являются включаемыми\footnote{Прим. перев.: здесь автор
использует термин sourcable, происходящий от bash'овской директивы +source+,
обеспечивающей включение в скрипт кода из внешнего файла. В классическом POSIX
shell это соответствует оператору-точке <<+.+>>. В отличие от прямого запуска
одного скрипта из другого, включаемый код исполняется той же самой оболочкой,
что и основной код, и при возвращении в основной скрипт сохраняются переменные
окружения, определенные во включаемом коде. Как правило, код для включения
не~содержит shebang'а (+#!/bin/sh+ в начале файла).} shell-скриптами, содержащими,
главным образом, определения переменных. Большинство файлов из этих каталогов
включаются в одноименные скрипты SysV init. Этот принцип отражен в
\href{http://www.debian.org/doc/debian-policy/ch-opersys.html#s-sysvinit}{Debian
Policy Manual (раздел 9.3.2)} и в
\href{http://fedoraproject.org/wiki/Packaging:SysVInitScript}{Fedora Packaging
Guidelines}, однако в обоих этих дистрибутивах иногда встречаются файлы,
не~соответствующие такой схеме, например, не~имеющие соответствующего
init-скрипта, или даже сами не~являющиеся скриптами.

Но почему вообще появились эти каталоги? Чтобы ответить на этот вопрос,
обратимся к истории развития концепции SysV init-скриптов. Исторически,
сложилось так, что они располагаются в каталоге под названием +/etc/rc.d/init.d+
(или что-то похожее).  Отметим, что каталог +/etc+ вообще-то предназначен для
хранения файлов конфигурации, а не~исполняемого кода (в частности, скриптов).
Однако, в начале своей истории, init-скрипты рассматривались именно как файлы
конфигурации, и редактирование их администратором было общепринятой практикой.
Но со временем, по мере роста и усложнения этих скриптов, их стали рассматривать
уже не~как файлы конфигурации, а как некие программы. Чтобы упростить их
настройку и обеспечить безопасность процесса обновления, настройки были вынесены
в отдельные файлы, загружаемые при работе init-скриптов.

Попробуем составить некоторое представление о настройках, которые можно сделать
через эти файлы. Вот краткий и неполный список различных параметров, которые
могут быть заданы через переменные окружения в таких файлах (составлен мною по
результатам исследования соответствующих каталогов в Fedora и Debian):
\begin{itemize}
	\item Дополнительные параметры командной строки для бинарника демона.
	\item Настройки локали для демона.
	\item Тайм-аут остановки для демона.
	\item Режим остановки для демона.
	\item Общесистемные настройки, например, системная локаль, часовой пояс,
		параметры клавиатуры для консоли.
	\item Избыточные информация о системных настройках, например, указание,
		установлены ли аппаратные часы по Гринвичу или по местному
		времени.
	\item Списки правил брандмауэра, не~являются скриптами (!).
	\item Привязка к процессорным ядрам для демона.
	\item Настройки, не~относящиеся к процессу загрузки, например,
		информация по установке пакетов с новыми ядрами, конфигурация
		nspluginwrapper, разрешение на выполнение
		предварительного связывания (prelinking) библиотек.
	\item Указание, нужно ли запускать данную службу или нет.
	\item Настройки сети.
	\item Перечень модулей ядра, которые должны быть подгружены
		принудительно.
	\item Нужно ли отключать питание компьютера при остановке системы
		(+poweroff+) или нет (+halt+).
	\item Права доступа для файлов устройств (!).
	\item Описание соответствующей SysV службы.
	\item Идентификатор пользователя/группы, значение umask для демона.
	\item Ограничения по ресурсам для демона.
	\item Приоритет OOM killer'а для демона.
\end{itemize}

А теперь давайте ответим на вопрос: что же такого неправильного в
+/etc/sysconfig+ (+/etc/default+) и почему этим каталогам нет места в мире
systemd?
\begin{itemize}
	\item Прежде всего, утрачены основная цель и смысл существования этих
		каталогов: файлы конфигурации юнитов systemd не~являются
		программами, в отличие от init-скриптов SysV. Эти файлы
		представляют собой простые, декларативные описания конкретных
		задач и функций, и обычно содержат не~более шести строк. Они
		легко могут быть сгенерированы и проанализованы без
		использования Bourne shell. Их легко читать и понимать. Кроме
		того, их легко модифицировать: достаточно скопировать их из
		+/lib/systemd/system+ в +/etc/systemd/system+, после чего внести
		необходимые изменения в скопированный файл (при этом можно быть
		уверенным, что изменения не~будут затерты пакетным менеджером).
		Изначальная причина появления обсуждаемых каталогов~---
		необходимость разделять код и параметры конфигурации~--- больше
		не~существует, так как файлы описания юнитов не~являются кодом.
		Проще говоря, обсуждаемые каталоги являются решением проблемы,
		которой уже не~существует.
	\item Обсуждаемые каталоги и файлы в них очень сильно привязаны к
		специфике дистрибутивов. Мы же планируем, используя systemd,
		способствовать стандартизации и унификации дистрибутивов. В
		частности, одним из факторов такой стандартизации является
		рекомендация распространять соответствующие файлы конфигурации
		юнитов сразу с апстримным продуктом, а не~возлагать эту работу
		на создателей пакетов, как это делалась во времена SysV.
		Так как расположение обсуждаемых каталогов и настраиваемые через
		них параметры сильно отличаются от дистрибутива к дистрибутиву,
		пытаться поддерживать их в апстримных файлах конфигурации юнитов
		просто бессмысленно. Хранение параметров конфигурации в этих  
		каталогах~--- один из факторов, превращающих Linux в зоопарк
		несовместимых решений.
	\item Большинство настроек, задаваемых через эти каталоги, являются
		избыточными в мире systemd. Например, различные службы позволяют
		задать таким методом параметры исполнения процесса, в частности,
		идентификатор пользователя/группы, ограничения ресурсов,
		привязки к ядрам CPU, приоритет OOM killer'а. Однако, эти
		настройки поддерживаются лишь некоторыми init-скриптами, причем
		одна и та же настройка в различных скриптах может называться
		по-разному. С другой стороны, в мире systemd, все эти настройки
		доступны для всех служб без исключения, и всегда задаются
		одинаково, через одни и те же параметры конфигурационных файлов.
	\item Файлы конфигурации юнитов имеют множество удобных и простых в
		использовании настроек среды исполнения процесса, гораздо
		больше, чем могут предоставить файлы из +/etc/sysconfig+.
	\item Необходимость в некоторых из этих настроек весьма сомнительна.
		Например, возьмем вышеупомянутую возможность задавать
		идентификатор пользователя/группы для процесса. Эту задачу
		должен решать разработчик ПО или дистрибутива. Вмешательство
		администратора в данную настройку, как правило, лишено
		смысла~--- только разработчик располагает всей информацией,
		позволяющий предотвратить конфликты идентификаторов и имен
		пользователей и групп.
	\item Формат файлов, используемых для сохранения настроек, плохо
		подходит для данной задачи. Так как эти файлы, как правило,
		являются включаемыми shell-скриптами, ошибки при их чтении очень
		трудно отследить. Например, ошибка в имени переменной приведет к
		тому, что переменная не~будет изменена, однако никакого
		предупреждения при этом не~выводится.
	\item Кроме того, такая организация не~исключает влияния
		конфигурационных параметров на среду исполнения: например,
		изменение переменных +IFS+ и +LANG+ может существенно повлиять
		на результат интерпретации init-скрипта.
	\item Интерпретация этих файлов требует запуска еще одного экземпляра
		оболочки, что приводит к задержкам при загрузке\footnote{Прим.
		перев.: Здесь автор несколько заблуждается. Скрипты, включенные
		через директиву +source+, исполняются тем же экземпляром
		оболочки, что и вызвавший их скрипт.}.
	\item Файлы из +/etc/sysconfig+ часто пытаются использовать в качестве
		суррогатной замены файлов конфигурации для тех демонов, которые
		не~имеют встроенной поддержки конфигурационных файлов. В
		частности, вводятся специальные переменные, позволяющие задать
		аргументы командной строки, используемые при запуске демона.
		Встроенная поддержка конфигурационных файлов является более
		удобной альтернативой такому подходу, ведь, глядя на ключи
		<<+-k+>>, <<+-a+>>, <<+-f+>>, трудно догадаться об их
		назначении. Очень часто, из-за ограниченности словаря, на
		различных демонов одни и те же ключи действуют совершенно
		по-разному (для одного демона ключ <<+-f+>> содержит указание
		демонизироваться при запуске, в то время как для другого эта
		опция действует прямо противоположным образом.) В отличие от
		конфигурационных файлов, строка запуска не~может включать
		полноценных комментариев.
	\item Некоторые из настроек, задаваемых в +/etc/sysconfig+, являются
		полностью избыточными. Например, во многие дистрибутивах
		подобным методом указывается, установлены ли аппаратные часы
		компьютера по Гринвичу, или по местному времени. Однако эта же
		настройка задается третьей строкой файла +/etc/adjtime+,
		поддерживаемого во всех дистрибутивах. Использование
		избыточного и не~стандартизированного параметра конфигурации
		только добавляет путаницу и не~несет никакой пользы.
	\item Многие файлы настроек из +/etc/sysconfig+ позволяют отключать
		запуск соответствующей службы. Однако эта операция уже
		поддерживается штатно для всех служб, через команды
		+systemctl enable+/+disable+ (или +chkconfig on+/+off+).
		Добавление дополнительного уровня настройки не~приносит никакой
		пользы и лишь усложняет работу администратора.
	\item Что касается списка принудительно загружаемых модулей ядра~--- в
		настоящее время существуют куда более удобные пути для
		автоматической подгрузки модулей при загрузке системы. Например,
		многие модули автоматически подгружаются +udev+'ом при
		обнаружении соответствующего оборудования. Этот же принцип
		распространяется на ACPI и другие высокоуровневые технологии.
		Одно из немногих исключений из этого правила~--- к сожалению, в
		настоящее время не~поддерживается автоматическая загрузка
		модулей на основании информации о возможностях процессора,
		однако это будет исправлено в ближайшем будущем. В случае, если
		нужный вам модуль ядра все же не~может быть подгружен
		автоматически, все равно существует гораздо более удобные методы
		указать его принудительную подгрузку~--- например, создав
		соответствующий файл в каталоге
		\hreftt{http://0pointer.de/public/systemd-man/modules-load.d.html}{/etc/modules-load.d/}
		(стандартный метод настройки принудительной подгрузки модулей).
	\item И наконец, хотелось бы отметить, что каталог +/etc+ определен как
		место для хранения системных настроек (<<Host-specific system
		configuration>>, согласно FHS). Наличие внутри него подкаталога
		+sysconfig+, который тоже содержит системную конфигурацию,
		является очевидно избыточным.
\end{itemize}

Что же можно предложить в качестве современной, совместимой с systemd
альтернативы настройке системы через файлы в этих каталогах? Ниже приведены
несколько рекомендаций, как лучше поступить с задаваемыми таким образом параметрами
конфигурации:
\begin{itemize}
	\item Попробуйте просто отказаться от них. Если они полностью избыточны (например,
		настройка аппаратных часов на Гринвич/местное время), то убрать
		их будет довольно легко (если не~рассматривать вопросы
		обеспечения совместимости). Если аналогичные по смыслу опции
		штатно поддерживаются systemd, нет никакого смысла дублировать
		их где-то еще (перечень опций, которые можно задать для любой
		службы, приведен на страницах справки
		\href{http://0pointer.de/public/systemd-man/systemd.exec.html}{systemd.exec(5)}
		и
		\href{http://0pointer.de/public/systemd-man/systemd.service.html}{systemd.service(5)}.)
		Если же ваша настройка просто добавляет еще один уровень
		отключения запуска службы~--- не~плодите лишние сущности,
		откажитесь от нее.
	\item Найдите для них более подходящее место. Например, в случае с
		некоторыми общесистемными настройками (такими, как локаль или
		часовой пояс), мы надеемся аккуратно подтолкнуть дистрибутивы в
		правильно направлении (см. предыдущий эпизод).
	\item Добавьте их поддержку в штатную систему настройки демона через
		собственные файлы конфигурации. К счастью, большинство служб,
		работающих в Linux, являются свободным программным обеспечением,
		так что сделать это довольно просто.
\end{itemize}

Существует лишь одна причина поддерживать эти файлы еще некоторое
время: необходимо обеспечить совместимость при обновлении. Тем не~менее, как
минимум в новых пакетах, от этих файлов лучше отказаться.

Если требование совместимости критично, вы можете задействовать эти
конфигурационные файлы даже в том случае, если настраиваете службы через
штатные unit-файлы systemd. Если ваш файл из +sysconfig+ содержит лишь
определения переменных, можно воспользоваться опцией
+EnvironmentFile=-/etc/sysconfig/foobar+ (подробнее об этой опции см.
\href{http://0pointer.de/public/systemd-man/systemd.exec.html}{systemd.exec(5)}),
позволяющей прочитать из файла набор переменных окружения, который будет
установлен при запуске службы. Если же для задания настроек вам необходим
полноценный язык программирования~--- ничто не~мешает им воспользоваться.
Например, вы можете создайть в +/usr/lib/<your package>/+ простой скрипт,
который включает соответствующие файлы, а затем запускает бинарник демона через
+exec+. После чего достаточно просто указать этот скрипт в опции +ExecStart=+
вместо бинарника демона.

\section{Экземпляры служб}

Большинство служб в Linux/Unix являются одиночными (singleton): в каждый момент
времени на данном хосте работает только один экземпляр службы. В качестве
примера таких одиночных служб можно привести Syslogd, Postfix, Apache. Однако,
существуют службы, запускающие по несколько экземпляров себя на одном хосте.
Например, службы наподобие Dovecot IMAP запускают по одному экземпляру на каждый
локальный порт и/или IP-адрес. Другой пример, который можно встретить
практически во всех системах~--- \emph{getty}, небольшая служба, запускающаяся на
каждом TTY (от +tty1+ до +tty6+). На некоторых серверах, в зависимости от
сделанных администратором настроек или параметров загрузки, могут запускаться
дополнительные экземпляры getty, для подключаемых к COM-портам терминалов или
для консоли системы виртуализации. Еще один пример службы, работающей в
нескольких экземплярах (по крайней мере, в мире systemd)~--- \emph{fsck},
программа проверки файловой системы, которая запускается по одному экземпляру
на каждое блочное устройство, требующее такой проверки. И наконец, стоит
упомянуть службы с активацией в стиле inetd~--- при обращении через сокет, по
одному экземпляру на каждое соединение. В этой статье я попытаюсь рассказать,
как в systemd реализовано управление <<многоэкземплярными>> службами, и какие
выгоды системный администратор может извлечь из этой возможности.

Если вы читали предыдущие статьи из этого цикла, вы, скорее всего, уже знаете,
что службы systemd именуются по схеме \emph{foobar}+.service+, где
\emph{foobar}~--- строка, идентифицирующая службу (проще говоря, ее имя), а
+.service+~--- суффикс, присутствующий в именах всех файлов конфигурации служб.
Сами эти файлы могут находиться в каталогах +/etc/systemd/systemd+ и
+/lib/systemd/system+ (а также, возможно, и в других). Для служб, работающих в
нескольких экземплярах, эта схема становится немного сложнее:
\emph{foobar}+@+\emph{quux}+.service+, где \emph{foobar}~--- имя службы,
общее для всех экземпляров, а \emph{quux}~--- идентификатор конкретного
экземпляра. Например, +serial-gett@ttyS2.service+~--- это служба getty для
COM-порта, запущенная на +ttyS2+.

При необходимости, экземпляры служб можно легко создать динамически. Скажем, вы
можете, безо всяких дополнительных настроек, запустить новый экземпляр getty на
последовательном порту, просто выполнив +systemctl start+ для нового экземпляра:
\begin{Verbatim}
# systemctl start serial-getty@ttyUSB0.service
\end{Verbatim}

Получив такую команду, systemd сначала пытается найти файл конфигурации юнита с
именем, точно соответствующим запрошенному. Если такой файл найти не~удается
(при работе с экземплярами сервисов обычно так и происходит), из имени файла
удаляется идентификатор экземпляра, и полученное имя используется при поиске
\emph{шаблона} конфигурации. В нашем случае, если отсутствует файл с именем
+serial-getty@ttyUSB0.service+, используется файл-шаблон под названием
+serial-getty@.service+. Таким образом, для всех экземпляров данной службы,
используется один и тот же шаблон конфигурации. В случае с getty для COM-портов,
этот шаблон, поставляемый в комплекте с systemd
(файл +/lib/systemd/system/serial-getty@.service+) выглядит примерно так:
\begin{Verbatim}
[Unit]
Description=Serial Getty on %I
BindTo=dev-%i.device
After=dev-%i.device systemd-user-sessions.service

[Service]
ExecStart=-/sbin/agetty -s %I 115200,38400,9600
Restart=always
RestartSec=0
\end{Verbatim}
(Заметим, что приведенная здесь версия немного сокращена, по сравнению с реально
используемой в systemd. Удалены не~относящиеся к теме нашего обсуждения
параметры конфигурации, обеспечивающие совместимость с SysV, очистку экрана и
удаление предыдущих пользователей с текущего TTY. Если вам интересно, можете
посмотреть
\href{http://cgit.freedesktop.org/systemd/plain/units/serial-getty@.service.m4}{полную
версию}.)

Этот файл похож на обычный файл конфигурации юнита, с единственным отличием: в
нем используются спецификаторы \%I и \%i. В момент загрузки юнита, systemd
заменяет эти спецификаторы на идентификатор экземпляра службы. В нашем случае,
при обращении к экземпляру +serial-getty@ttyUSB0.service+, они заменяются на
<<+ttyUSB0+>>. Результат этой замены можно проверить, например, запросив
состояние для этой службы:
\begin{Verbatim}[commandchars=\\\{\}]
$ systemctl status serial-getty@ttyUSB0.service
serial-getty@ttyUSB0.service - Getty on ttyUSB0
	  Loaded: loaded (/lib/systemd/system/serial-getty@.service; static)
	  Active: active (running) since Mon, 26 Sep 2011 04:20:44 +0200; 2s ago
	Main PID: 5443 (agetty)
	  CGroup: name=systemd:/system/getty@.service/ttyUSB0
		  \llangl{} 5443 /sbin/agetty -s ttyUSB0 115200,38400,9600
\end{Verbatim}
Собственно, это и есть ключевая идея организации экземпляров служб. Как видите,
systemd предоставляет простой в использовании механизм шаблонов, позволяющих
динамически создавать экземпляры служб. Добавим несколько дополнительных
замечаний, позволяющих эффективно использовать этот механизм:

Вы можете создавать дополнительные экземпляры таких служб, просто добавляя
символьные ссылки в каталоги +*.wants/+. Например, чтобы обеспечить запуск getty
на ttyUSB0 при каждой загрузке, достаточно создать такую ссылку:
\begin{Verbatim}
# ln -s /lib/systemd/system/serial-getty@.service \
/etc/systemd/system/getty.target.wants/serial-getty@ttyUSB0.service
\end{Verbatim}
При этом файл конфигурации, на который указывает ссылка (в нашем случае
+serial-getty@.service+), будет вызван с тем именем экземпляра, которое указанно
в названии этой ссылки (в нашем случае~--- +ttyUSB0+).

Вы не~сможете обратиться к юниту-шаблону без указания идентификатора экземпляра.
В частности, команда +systemctl start serial-getty@.service+ завершится ошибкой.

Иногда возникает необходимость отказаться от использования общего шаблона
для конкретного экземпляра (т.е. конфигурация данного экземпляра настолько
сильно отличается от конфигурации остальных экземпляров данной службы, что
механизм шаблонов оказывается неэффективен). Специально для таких случаев, в
systemd и заложен предварительный поиск файла с именем, точно соответствующим
указанному (прежде чем использовать общий шаблон). Таким образом, вы можете
поместить файл с именем, точно соответствующим полному титулу экземпляра, в
каталог +/etc/systemd/system/+~--- и содержимое этого файла, при обращении
к выбранному экземпляру, полностью перекроет все настройки, сделанные в общем
шаблоне.

В приведенном выше файле, в некоторых местах используется спецификатор +%I+, а
в других~--- +%i+. У вас может возникнуть закономерный вопрос~--- чем они
отличаются? +%i+ всегда точно соответствует идентификатору экземпляра, в то
время, как +%I+ соответствует экранированной (escaped) версии этого
идентификатора. Когда идентификатор не~содержит спецсимволов (например,
+ttyUSB0+). Но имена устройств, например, содержат слеши (<</>>), которые
не~могут присутствовать в имени юнита (и в имени файла на Unix). Поэтому, перед
использованием такого имени в качестве идентификатора устройства, оно должно
быть экранировано~--- <</>> заменяются на <<->>, а большинство других
специальных символов (включая <<->>) заменяются последовательностями вида
+\xAB+, где AB~--- ASCII-код символа, записанный в шестнадцатеричной системе
счисления\footnote{Согласен, этот алгоритм дает на выходе не~очень читабельный
результат. Но этим грешат практически все алгоритмы экранирования. В данном
случае, были использован механизм экранирования из udev, с одним изменением. В
конце концов, нам нужно было выбрать что-то. Если вы собираетесь комментировать
наш алгоритм экранирования~--- пожалуйста, также укажите, где вы живете, чтобы я
мог приехать к вам и раскрасить ваш велосипедный гараж в синий с желтыми
полосами. Спасибо!}\,\footnote{Прим. перев.: В предыдущем примечании автор снова
намекает на Паркинсовский Закон Тривиальности. Действительно, выбор алгоритма
экранирования практически не~влияет ни на работу systemd, ни на управлением им
(файлы/юниты с экранированными именами, как правило либо создаются специальными
программами-генераторами, либо формируются systemd на лету; при выводе состояния
показываются и неэкранированные имена; при вводе команд, как обычно, на
помощь приходит автодополнение оболочки).}. Например, чтобы обратиться
последовательному USB-порту по его адресу на шине, нам нужно использовать имя
наподобие +serial/by-path/pci-0000:00:1d.0-usb-0:1.4:1.1-port0+. Экранированная
версия этого имени~---
+serial-by\x2dpath-pci\x2d0000:00:1d.0\x2dusb\x2d0:1.4:1.1\x2dport0+. +%I+ будет
ссылаться на первую из этих строк, +%i+~--- на вторую. С практической точки
зрения, это означает, что спецификатор +%i+ можно использовать в том случае,
когда надо сослаться на имена других юнитов, например, чтобы описать
дополнительные зависимости (в случае с +serial-getty@.service+, этот
спецификатор используется для ссылки на юнит +dev-%i.device+, соответствующий
одноименному устройству). В то время как +%I+ удобно использовать в командной
строке (+ExecStart+) и для формирования читабельных строк описания. Рассмотрим
работу этих принципов на примере нашего юнит-файла:
\begin{landscape}
\begin{Verbatim}[fontsize=\small]
# systemctl start 'serial-getty@serial-by\x2dpath-pci\x2d0000:00:1d.0\x2dusb\x2d0:1.4:1.1\x2dport0.service'
# systemctl status 'serial-getty@serial-by\x2dpath-pci\x2d0000:00:1d.0\x2dusb\x2d0:1.4:1.1\x2dport0.service'
serial-getty@serial-by\x2dpath-pci\x2d0000:00:1d.0\x2dusb\x2d0:1.4:1.1\x2dport0.service - Serial Getty on serial/by-path/pci-0000:00:1d.0-usb-0:1.4:1.1-port0
	  Loaded: loaded (/lib/systemd/system/serial-getty@.service; static)
	  Active: active (running) since Mon, 26 Sep 2011 05:08:52 +0200; 1s ago
	Main PID: 5788 (agetty)
	  CGroup: name=systemd:/system/serial-getty@.service/serial-by\x2dpath-pci\x2d0000:00:1d.0\x2dusb\x2d0:1.4:1.1\x2dport0
		    5788 /sbin/agetty -s serial/by-path/pci-0000:00:1d.0-usb-0:1.4:1.1-port0 115200 38400 9600
\end{Verbatim}
\end{landscape}
Как видите, в качестве идентификатора экземпляра используется экранированная
версия, в то время как в строке описания и в командной строке +getty+
используется неэкранированная версия. Как и предполагалось.

(Небольшое замечание: помимо +%i+ и +%I+, существует еще несколько
спецификаторов, и большинство из них доступно и в обычных файлах конфигурации
юнитов, а не~только в шаблонах. Подробности можно посмотреть на
\href{http://0pointer.de/public/systemd-man/systemd.unit.html}{странице
руководства}, содержащей полный перечень этих спецификаторов с краткими
пояснениями.)

\end{document}

vim:ft=tex:tw=80:spell:spelllang=ru
