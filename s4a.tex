\documentclass[10pt,oneside,a4paper]{article}
\usepackage{cmap} % Copy-paste из PDF без проблем с кодировкой
\usepackage[utf8]{inputenc}
\usepackage[english,russian]{babel} % Русские переносы и проч.
\usepackage[pdftex]{graphicx,color}
\usepackage[T1,T2A]{fontenc}
\usepackage{indentfirst} % Отступ в первом абзаце главы 
\usepackage{fancyvrb} % Продвинутые листинги и in-line commands
% listings для данной ситуации, имхо, избыточен
\usepackage{pdflscape}
\usepackage[colorlinks,unicode,urlcolor=blue]{hyperref}
% Заполняем поля PDF уже со включенной опцией unicode
\hypersetup{pdftitle={systemd для администраторов},%
pdfauthor={Lennart Poettering, Sergey Ptashnick}}
% Небольшое сокращение
\newcommand{\sectiona}[1]{\section*{#1}\addcontentsline{toc}{section}{#1}}
% Настройка макета страницы
\setlength{\hoffset}{-1.5cm}
\addtolength{\textwidth}{2cm}
\setlength{\voffset}{-1.5cm}
\addtolength{\textheight}{2cm}
% Настройка in-line commands
\DefineShortVerb{\+}
\VerbatimFootnotes
% И листингов
\definecolor{gray}{gray}{0.75}
\fvset{frame=leftline,rulecolor=\color{gray},framerule=1mm}
% Запрет висячих строк
\clubpenalty=10000
\widowpenalty=10000

\begin{document}
\sloppy
\title{systemd для администраторов}
\author{Lennart P\"{o}ttering (author)\\Sergey Ptashnick (russian translation)\\%
\small Licensed under
\href{http://creativecommons.org/licenses/by-sa/3.0/legalcode}{CC-BY-SA}}
\maketitle
\tableofcontents%\newpage
\sectiona{Предисловие автора}
Многие из вас, наверное, уже знают, что
\href{http://www.freedesktop.org/wiki/Software/systemd}{systemd}~--- это
новая система инициализации дистрибутива Fedora, начиная с
Fedora~14\footnote{Прим. перев.: к сожалению, разработчики Fedora приняли
решение оставить в Fedora~14 в качестве системы инициалиазции по умолчанию
upstart, однако systemd все равно будет включен в этот релиз и может быть
использован в качестве альтернативной системы инициализации}. Помимо Fedora,
systemd также поддерживает и другие дистрибутивы, в частности,
\href{http://en.opensuse.org/SDB:Systemd}{OpenSUSE}.
systemd предоставляет администраторам целый ряд новых возможностей,
значительно упрощающих процесс обслуживания системы. Эта статья является
первой в серии публикаций, планируемых в ближайшие месяцы. В каждой из этих
статей я попытаюсь рассказать об очередной новой возможности systemd.
Большинство этих возможностей можно описать легко и просто, и подобные статьи
должны быть интересны довольно широкой аудитории.  Однако, время от времени
мы будем рассматривать ключевые новшества systemd, что может потребовать
несколько более подробного изложения. 
\begin{flushright}
	Lennart P\"{o}ttering, 23 августа 2010
\end{flushright}

\section{Контроль процесса загрузки}
Как правило, во время загрузки Linux по экрану быстро пробегает огромное
количество различных сообщений. Так как мы интенсивно работаем над
параллелизацией и ускорением процесса загрузки, с каждой новой верисей
systemd эти сообщения будут пробегать все быстрее и быстрее, вследствие чего,
читать их будет все труднее. К тому же, многие пользователи применяют
графические оболочки загрузки (например, Plymouth), полностью скрывающие эти
сообщения. Тем не~менее, информация, которую несут эти сообщения, была и
остается чрезвычайно важной — они показывают, успешно ли запустилась каждая
служба, или попытка ее запуска закончилась ошибкой (зеленое
\texttt{[~\textcolor{green}{OK}~]} или красное
\texttt{[~\textcolor{red}{FAILED}~]} соответственно). Итак, с ростом скорости
загрузки систем, возникает неприятная ситуация: информация о результатах
запуска служб бывает очень важна, а просматривать ее все тяжелее. systemd
предлагает выход из этой ситуации: он отслеживает и запоминает факты успешного
или неудачного запуска служб на этапе загрузки, а такжи сбои служб во время
работы. К таким случая относятся выходы с ненулевым кодом, ошибки
сегментирования и т.п. Введя systemctl status в своей командной оболочке, вы
можете ознакомиться с состоянием всех служб, как <<родных>> (native) для
systemd, так и классических SysV/LSB служб, поддерживаемых в целях
совместимости: 

\begin{landscape}
\begin{Verbatim}[fontsize=\small]
[root@lambda] ~# systemctl
UNIT                                          LOAD   ACTIVE       SUB        JOB       DESCRIPTION
dev-hugepages.automount                       loaded active       running              Huge Pages File System Automount Point
dev-mqueue.automount                          loaded active       running              POSIX Message Queue File System Automount Point
proc-sys-fs-binfmt_misc.automount             loaded active       waiting              Arbitrary Executable File Formats File System Automount Point
sys-kernel-debug.automount                    loaded active       waiting              Debug File System Automount Point
sys-kernel-security.automount                 loaded active       waiting              Security File System Automount Point
sys-devices-pc...0000:02:00.0-net-eth0.device loaded active       plugged              82573L Gigabit Ethernet Controller
[...]
sys-devices-virtual-tty-tty9.device           loaded active       plugged              /sys/devices/virtual/tty/tty9
-.mount                                       loaded active       mounted              /
boot.mount                                    loaded active       mounted              /boot
dev-hugepages.mount                           loaded active       mounted              Huge Pages File System
dev-mqueue.mount                              loaded active       mounted              POSIX Message Queue File System
home.mount                                    loaded active       mounted              /home
proc-sys-fs-binfmt_misc.mount                 loaded active       mounted              Arbitrary Executable File Formats File System
abrtd.service                                 loaded active       running              ABRT Automated Bug Reporting Tool
accounts-daemon.service                       loaded active       running              Accounts Service
acpid.service                                 loaded active       running              ACPI Event Daemon
atd.service                                   loaded active       running              Execution Queue Daemon
auditd.service                                loaded active       running              Security Auditing Service
avahi-daemon.service                          loaded active       running              Avahi mDNS/DNS-SD Stack
bluetooth.service                             loaded active       running              Bluetooth Manager
console-kit-daemon.service                    loaded active       running              Console Manager
cpuspeed.service                              loaded active       exited               LSB: processor frequency scaling support
crond.service                                 loaded active       running              Command Scheduler
cups.service                                  loaded active       running              CUPS Printing Service
dbus.service                                  loaded active       running              D-Bus System Message Bus
getty@tty2.service                            loaded active       running              Getty on tty2
getty@tty3.service                            loaded active       running              Getty on tty3
getty@tty4.service                            loaded active       running              Getty on tty4
getty@tty5.service                            loaded active       running              Getty on tty5
getty@tty6.service                            loaded active       running              Getty on tty6
haldaemon.service                             loaded active       running              Hardware Manager
hdapsd@sda.service                            loaded active       running              sda shock protection daemon
irqbalance.service                            loaded active       running              LSB: start and stop irqbalance daemon
iscsi.service                                 loaded active       exited               LSB: Starts and stops login and scanning of iSCSI devices.
iscsid.service                                loaded active       exited               LSB: Starts and stops login iSCSI daemon.
livesys-late.service                          loaded active       exited               LSB: Late init script for live image.
livesys.service                               loaded active       exited               LSB: Init script for live image.
lvm2-monitor.service                          loaded active       exited               LSB: Monitoring of LVM2 mirrors, snapshots etc. using dmeventd or progress polling
mdmonitor.service                             loaded active       running              LSB: Start and stop the MD software RAID monitor
modem-manager.service                         loaded active       running              Modem Manager
netfs.service                                 loaded active       exited               LSB: Mount and unmount network filesystems.
NetworkManager.service                        loaded active       running              Network Manager
ntpd.service                                  loaded maintenance  maintenance          Network Time Service
polkitd.service                               loaded active       running              Policy Manager
prefdm.service                                loaded active       running              Display Manager
rc-local.service                              loaded active       exited               /etc/rc.local Compatibility
rpcbind.service                               loaded active       running              RPC Portmapper Service
rsyslog.service                               loaded active       running              System Logging Service
rtkit-daemon.service                          loaded active       running              RealtimeKit Scheduling Policy Service
sendmail.service                              loaded active       running              LSB: start and stop sendmail
sshd@172.31.0.53:22-172.31.0.4:36368.service  loaded active       running              SSH Per-Connection Server
sysinit.service                               loaded active       running              System Initialization
systemd-logger.service                        loaded active       running              systemd Logging Daemon
udev-post.service                             loaded active       exited               LSB: Moves the generated persistent udev rules to /etc/udev/rules.d
udisks.service                                loaded active       running              Disk Manager
upowerd.service                               loaded active       running              Power Manager
wpa_supplicant.service                        loaded active       running              Wi-Fi Security Service
avahi-daemon.socket                           loaded active       listening            Avahi mDNS/DNS-SD Stack Activation Socket
cups.socket                                   loaded active       listening            CUPS Printing Service Sockets
dbus.socket                                   loaded active       running              dbus.socket
rpcbind.socket                                loaded active       listening            RPC Portmapper Socket
sshd.socket                                   loaded active       listening            sshd.socket
systemd-initctl.socket                        loaded active       listening            systemd /dev/initctl Compatibility Socket
systemd-logger.socket                         loaded active       running              systemd Logging Socket
systemd-shutdownd.socket                      loaded active       listening            systemd Delayed Shutdown Socket
dev-disk-by\x1...x1db22a\x1d870f1adf2732.swap loaded active       active               /dev/disk/by-uuid/fd626ef7-34a4-4958-b22a-870f1adf2732
basic.target                                  loaded active       active               Basic System
bluetooth.target                              loaded active       active               Bluetooth
dbus.target                                   loaded active       active               D-Bus
getty.target                                  loaded active       active               Login Prompts
graphical.target                              loaded active       active               Graphical Interface
local-fs.target                               loaded active       active               Local File Systems
multi-user.target                             loaded active       active               Multi-User
network.target                                loaded active       active               Network
remote-fs.target                              loaded active       active               Remote File Systems
sockets.target                                loaded active       active               Sockets
swap.target                                   loaded active       active               Swap
sysinit.target                                loaded active       active               System Initialization

LOAD   = Reflects whether the unit definition was properly loaded.
ACTIVE = The high-level unit activation state, i.e. generalization of SUB.
SUB    = The low-level unit activation state, values depend on unit type.
JOB    = Pending job for the unit.

221 units listed. Pass --all to see inactive units, too.
[root@lambda] ~#
\end{Verbatim}
(Листинг был сокращен за счет удаления строк, не~относящихся к теме статьи.)
\end{landscape}

Обратите внимание на графу ACTIVE, в которой отображается обобщенный статус
службы (или любого другого юнита systemd: устройства, сокета, точки
монтирования~--- их мы рассмотрим подробнее в последующих статьях). Основными
значениями обощенного статуса являются active (служба выполняется) и inactive
(служба не~была запущена). Также существуют и другие статусы. Например,
внимательно посмотрев на листинг выше, вы можете заметить, что служба ntpd
(сервер точного времени) находится в состоянии, обозначенном как maintenance.
Чтобы узнать, что же произошло с ntpd, воспользуемся командой
+systemctl status+: 
\begin{Verbatim}[commandchars=\\\{\}]
[root@lambda] ~# systemctl status ntpd.service
ntpd.service - Network Time Service
	  Loaded: loaded (/etc/systemd/system/ntpd.service)
	  Active: \textcolor{red}{maintenance}
	    Main: 953 (code=exited, status=255)
	  CGroup: name=systemd:/systemd-1/ntpd.service
[root@lambda] ~#
\end{Verbatim}

systemd сообщает нам, что ntpd был запущен (с идентификатором процесса 953) и
аварийно завершил работу (с кодом выхода 255).

В последующих версиях systemd, мы планируем добавить возможность вызова в
таких ситуациях ABRT (Automated Bug Report Tool), но для этого необходима
соответствующая поддержка со стороны самого ABRT. Соответствующий запрос уже
\href{https://bugzilla.redhat.com/show_bug.cgi?id=622773}{направлен} его
разработчикам, однако пока не~встретил среди них поддержки.

Резюме: использование +systemctl+ и +systemctl status+ представляет современную,
более удобную и эффективную альтернативу разглядыванию быстро пробегающих по
экрану сообщений в классическом SysV. +systemctl status+ дает возможность
получить развернутую информацию о характере ошибки и, кроме того, в отличие
от сообщений SysV, показывает не~только ошибки при запуске, но и ошибки,
возникшие во время исполнения службы. 

\section{О службах и процессах}
В большинстве современных Linux-систем количество одновременно работающих
процессов обычно весьма значительно. Понять, откуда взялся и что делает тот
или иной процесс, становится все сложнее и сложнее. Многие службы используют
сразу несколько рабочих процессов, и это отнюдь не~всегда можно легко
распознать по выводу команды ps. Встречаются еще более сложные ситуации,
когда демон запускает сторонние процессы~--- например, веб-сервер выполняет
CGI-программы, а демон cron~--- команды, предписанные ему в crontab.

Немного помочь в решении этой проблемы может древовидная иерархия процессов,
отображаемая по команде +ps xaf+. Именно <<немного помочь>>, а не~решить
полностью. В частности, процессы, родители которых умирают раньше их самих,
становят потомками PID~1 (процесса init), что сразу затрудняет процесс
выяснения их происхождения. Кроме того, процесс может избавиться от связи с
родителем через две последовательные операции +fork()+ (В целом, эта возможность
признается нужной и полезной, и является частью используемого в Unix подхода
к разработке демонов.) Также, не~будем забывать, что процесс легко может
изменить свое имя посредством +PR_SETNAME+, или задав значение
+argv[0]+, что также усложняет процесс его опознания\footnote{Прим.
перев.: стоит отметить, что перечисленные ситуации могут возникнуть не~только
вследствие ошибок в коде и/или конфигурации программ, но и в результате злого
умысла. Например, очень часто встречается ситуация, когда установленный на
взломанном сервере процесс-бэкдор маскируется под нормального демона, меняя
себе имя, скажем, на httpd}.

systemd предоставляет простой путь для решения обсуждаемой задачи. Запуская
очередной новый процесс, systemd помещает его в отдельную контрольную группу
с соответствующим именем. Контрольные группы Linux предоставляют очень
удобный инструмент для иерархической структуризации процессов: когда
какой-либо процесс порождает потомка, этот потомок автоматически включается в
ту же группу, что и родитель. При этом, что очень важно, непривилегированные
процессы не~могут изменить свое положение в этой иерархии. Таким образом,
контрольные группы позволяют точно установить происхождение конкретного
процесса, вне зависимости от того, сколько раз он форкался и переименовывал
себя~--- имя его контрольной группы невозможно спрятать или изменить. Кроме
того, при штатном завершении родительской службы, будут завершены и все
порожденные ею процессы, как бы они ни пытались сбежать. С systemd уже
невозможна ситуация, когда после остановки web-сервера, некорректно
форкнувшийся CGI-процесс продолжает исполняться вплоть до последних секунд
работы системы.

В этой статье мы рассмотрим две простых команды, которые позволят вам
наглядно увидеть схему взаимоотношений systemd и порожденных им процессов.
Первая из этих команд~--- все та же +ps+, однако на этот раз в ее параметры
добавлено указание выводить сведения по контрольным группам, а также другую
интересную информацию: 

\begin{landscape}
\begin{Verbatim}[fontsize=\small]
$ ps xawf -eo pid,user,cgroup,args
  PID USER     CGROUP                              COMMAND
    2 root     -                                   [kthreadd]
    3 root     -                                    \_ [ksoftirqd/0]
[...]
 4281 root     -                                    \_ [flush-8:0]
    1 root     name=systemd:/systemd-1             /sbin/init
  455 root     name=systemd:/systemd-1/sysinit.service /sbin/udevd -d
28188 root     name=systemd:/systemd-1/sysinit.service  \_ /sbin/udevd -d
28191 root     name=systemd:/systemd-1/sysinit.service  \_ /sbin/udevd -d
 1096 dbus     name=systemd:/systemd-1/dbus.service /bin/dbus-daemon --system --address=systemd: --nofork --systemd-activation
 1131 root     name=systemd:/systemd-1/auditd.service auditd
 1133 root     name=systemd:/systemd-1/auditd.service  \_ /sbin/audispd
 1135 root     name=systemd:/systemd-1/auditd.service      \_ /usr/sbin/sedispatch
 1171 root     name=systemd:/systemd-1/NetworkManager.service /usr/sbin/NetworkManager --no-daemon
 4028 root     name=systemd:/systemd-1/NetworkManager.service  \_ /sbin/dhclient -d -4 -sf /usr/libexec/nm-dhcp-client.action -pf /var/run/dhclient-wlan0.pid -lf /var/lib/dhclient/dhclient-7d32a784-ede9-4cf6-9ee3-60edc0bce5ff-wlan0.lease -
 1175 avahi    name=systemd:/systemd-1/avahi-daemon.service avahi-daemon: running [epsilon.local]
 1194 avahi    name=systemd:/systemd-1/avahi-daemon.service  \_ avahi-daemon: chroot helper
 1193 root     name=systemd:/systemd-1/rsyslog.service /sbin/rsyslogd -c 4
 1195 root     name=systemd:/systemd-1/cups.service cupsd -C /etc/cups/cupsd.conf
 1207 root     name=systemd:/systemd-1/mdmonitor.service mdadm --monitor --scan -f --pid-file=/var/run/mdadm/mdadm.pid
 1210 root     name=systemd:/systemd-1/irqbalance.service irqbalance
 1216 root     name=systemd:/systemd-1/dbus.service /usr/sbin/modem-manager
 1219 root     name=systemd:/systemd-1/dbus.service /usr/libexec/polkit-1/polkitd
 1242 root     name=systemd:/systemd-1/dbus.service /usr/sbin/wpa_supplicant -c /etc/wpa_supplicant/wpa_supplicant.conf -B -u -f /var/log/wpa_supplicant.log -P /var/run/wpa_supplicant.pid
 1249 68       name=systemd:/systemd-1/haldaemon.service hald
 1250 root     name=systemd:/systemd-1/haldaemon.service  \_ hald-runner
 1273 root     name=systemd:/systemd-1/haldaemon.service      \_ hald-addon-input: Listening on /dev/input/event3 /dev/input/event9 /dev/input/event1 /dev/input/event7 /dev/input/event2 /dev/input/event0 /dev/input/event8
 1275 root     name=systemd:/systemd-1/haldaemon.service      \_ /usr/libexec/hald-addon-rfkill-killswitch
 1284 root     name=systemd:/systemd-1/haldaemon.service      \_ /usr/libexec/hald-addon-leds
 1285 root     name=systemd:/systemd-1/haldaemon.service      \_ /usr/libexec/hald-addon-generic-backlight
 1287 68       name=systemd:/systemd-1/haldaemon.service      \_ /usr/libexec/hald-addon-acpi
 1317 root     name=systemd:/systemd-1/abrtd.service /usr/sbin/abrtd -d -s
 1332 root     name=systemd:/systemd-1/getty@.service/tty2 /sbin/mingetty tty2
 1339 root     name=systemd:/systemd-1/getty@.service/tty3 /sbin/mingetty tty3
 1342 root     name=systemd:/systemd-1/getty@.service/tty5 /sbin/mingetty tty5
 1343 root     name=systemd:/systemd-1/getty@.service/tty4 /sbin/mingetty tty4
 1344 root     name=systemd:/systemd-1/crond.service crond
 1346 root     name=systemd:/systemd-1/getty@.service/tty6 /sbin/mingetty tty6
 1362 root     name=systemd:/systemd-1/sshd.service /usr/sbin/sshd
 1376 root     name=systemd:/systemd-1/prefdm.service /usr/sbin/gdm-binary -nodaemon
 1391 root     name=systemd:/systemd-1/prefdm.service  \_ /usr/libexec/gdm-simple-slave --display-id /org/gnome/DisplayManager/Display1 --force-active-vt
 1394 root     name=systemd:/systemd-1/prefdm.service      \_ /usr/bin/Xorg :0 -nr -verbose -auth /var/run/gdm/auth-for-gdm-f2KUOh/database -nolisten tcp vt1
 1495 root     name=systemd:/user/lennart/1             \_ pam: gdm-password
 1521 lennart  name=systemd:/user/lennart/1                 \_ gnome-session
 1621 lennart  name=systemd:/user/lennart/1                     \_ metacity
 1635 lennart  name=systemd:/user/lennart/1                     \_ gnome-panel
 1638 lennart  name=systemd:/user/lennart/1                     \_ nautilus
 1640 lennart  name=systemd:/user/lennart/1                     \_ /usr/libexec/polkit-gnome-authentication-agent-1
 1641 lennart  name=systemd:/user/lennart/1                     \_ /usr/bin/seapplet
 1644 lennart  name=systemd:/user/lennart/1                     \_ gnome-volume-control-applet
 1646 lennart  name=systemd:/user/lennart/1                     \_ /usr/sbin/restorecond -u
 1652 lennart  name=systemd:/user/lennart/1                     \_ /usr/bin/devilspie
 1662 lennart  name=systemd:/user/lennart/1                     \_ nm-applet --sm-disable
 1664 lennart  name=systemd:/user/lennart/1                     \_ gnome-power-manager
 1665 lennart  name=systemd:/user/lennart/1                     \_ /usr/libexec/gdu-notification-daemon
 1670 lennart  name=systemd:/user/lennart/1                     \_ /usr/libexec/evolution/2.32/evolution-alarm-notify
 1672 lennart  name=systemd:/user/lennart/1                     \_ /usr/bin/python /usr/share/system-config-printer/applet.py
 1674 lennart  name=systemd:/user/lennart/1                     \_ /usr/lib64/deja-dup/deja-dup-monitor
 1675 lennart  name=systemd:/user/lennart/1                     \_ abrt-applet
 1677 lennart  name=systemd:/user/lennart/1                     \_ bluetooth-applet
 1678 lennart  name=systemd:/user/lennart/1                     \_ gpk-update-icon
 1408 root     name=systemd:/systemd-1/console-kit-daemon.service /usr/sbin/console-kit-daemon --no-daemon
 1419 gdm      name=systemd:/systemd-1/prefdm.service /usr/bin/dbus-launch --exit-with-session
 1453 root     name=systemd:/systemd-1/dbus.service /usr/libexec/upowerd
 1473 rtkit    name=systemd:/systemd-1/rtkit-daemon.service /usr/libexec/rtkit-daemon
 1496 root     name=systemd:/systemd-1/accounts-daemon.service /usr/libexec/accounts-daemon
 1499 root     name=systemd:/systemd-1/systemd-logger.service /lib/systemd/systemd-logger
 1511 lennart  name=systemd:/systemd-1/prefdm.service /usr/bin/gnome-keyring-daemon --daemonize --login
 1534 lennart  name=systemd:/user/lennart/1        dbus-launch --sh-syntax --exit-with-session
 1535 lennart  name=systemd:/user/lennart/1        /bin/dbus-daemon --fork --print-pid 5 --print-address 7 --session
 1603 lennart  name=systemd:/user/lennart/1        /usr/libexec/gconfd-2
 1612 lennart  name=systemd:/user/lennart/1        /usr/libexec/gnome-settings-daemon
 1615 lennart  name=systemd:/user/lennart/1        /usr/libexec/gvfsd
 1626 lennart  name=systemd:/user/lennart/1        /usr/libexec//gvfs-fuse-daemon /home/lennart/.gvfs
 1634 lennart  name=systemd:/user/lennart/1        /usr/bin/pulseaudio --start --log-target=syslog
 1649 lennart  name=systemd:/user/lennart/1         \_ /usr/libexec/pulse/gconf-helper
 1645 lennart  name=systemd:/user/lennart/1        /usr/libexec/bonobo-activation-server --ac-activate --ior-output-fd=24
 1668 lennart  name=systemd:/user/lennart/1        /usr/libexec/im-settings-daemon
 1701 lennart  name=systemd:/user/lennart/1        /usr/libexec/gvfs-gdu-volume-monitor
 1707 lennart  name=systemd:/user/lennart/1        /usr/bin/gnote --panel-applet --oaf-activate-iid=OAFIID:GnoteApplet_Factory --oaf-ior-fd=22
 1725 lennart  name=systemd:/user/lennart/1        /usr/libexec/clock-applet
 1727 lennart  name=systemd:/user/lennart/1        /usr/libexec/wnck-applet
 1729 lennart  name=systemd:/user/lennart/1        /usr/libexec/notification-area-applet
 1733 root     name=systemd:/systemd-1/dbus.service /usr/libexec/udisks-daemon
 1747 root     name=systemd:/systemd-1/dbus.service  \_ udisks-daemon: polling /dev/sr0
 1759 lennart  name=systemd:/user/lennart/1        gnome-screensaver
 1780 lennart  name=systemd:/user/lennart/1        /usr/libexec/gvfsd-trash --spawner :1.9 /org/gtk/gvfs/exec_spaw/0
 1864 lennart  name=systemd:/user/lennart/1        /usr/libexec/gvfs-afc-volume-monitor
 1874 lennart  name=systemd:/user/lennart/1        /usr/libexec/gconf-im-settings-daemon
 1903 lennart  name=systemd:/user/lennart/1        /usr/libexec/gvfsd-burn --spawner :1.9 /org/gtk/gvfs/exec_spaw/1
 1909 lennart  name=systemd:/user/lennart/1        gnome-terminal
 1913 lennart  name=systemd:/user/lennart/1         \_ gnome-pty-helper
 1914 lennart  name=systemd:/user/lennart/1         \_ bash
29231 lennart  name=systemd:/user/lennart/1         |   \_ ssh tango
 2221 lennart  name=systemd:/user/lennart/1         \_ bash
 4193 lennart  name=systemd:/user/lennart/1         |   \_ ssh tango
 2461 lennart  name=systemd:/user/lennart/1         \_ bash
29219 lennart  name=systemd:/user/lennart/1         |   \_ emacs systemd-for-admins-1.txt
15113 lennart  name=systemd:/user/lennart/1         \_ bash
27251 lennart  name=systemd:/user/lennart/1             \_ empathy
29504 lennart  name=systemd:/user/lennart/1             \_ ps xawf -eo pid,user,cgroup,args
 1968 lennart  name=systemd:/user/lennart/1        ssh-agent
 1994 lennart  name=systemd:/user/lennart/1        gpg-agent --daemon --write-env-file
18679 lennart  name=systemd:/user/lennart/1        /bin/sh /usr/lib64/firefox-3.6/run-mozilla.sh /usr/lib64/firefox-3.6/firefox
18741 lennart  name=systemd:/user/lennart/1         \_ /usr/lib64/firefox-3.6/firefox
28900 lennart  name=systemd:/user/lennart/1             \_ /usr/lib64/nspluginwrapper/npviewer.bin --plugin /usr/lib64/mozilla/plugins/libflashplayer.so --connection /org/wrapper/NSPlugins/libflashplayer.so/18741-6
 4016 root     name=systemd:/systemd-1/sysinit.service /usr/sbin/bluetoothd --udev
 4094 smmsp    name=systemd:/systemd-1/sendmail.service sendmail: Queue runner@01:00:00 for /var/spool/clientmqueue
 4096 root     name=systemd:/systemd-1/sendmail.service sendmail: accepting connections
 4112 ntp      name=systemd:/systemd-1/ntpd.service /usr/sbin/ntpd -n -u ntp:ntp -g
27262 lennart  name=systemd:/user/lennart/1        /usr/libexec/mission-control-5
27265 lennart  name=systemd:/user/lennart/1        /usr/libexec/telepathy-haze
27268 lennart  name=systemd:/user/lennart/1        /usr/libexec/telepathy-logger
27270 lennart  name=systemd:/user/lennart/1        /usr/libexec/dconf-service
27280 lennart  name=systemd:/user/lennart/1        /usr/libexec/notification-daemon
27284 lennart  name=systemd:/user/lennart/1        /usr/libexec/telepathy-gabble
27285 lennart  name=systemd:/user/lennart/1        /usr/libexec/telepathy-salut
27297 lennart  name=systemd:/user/lennart/1        /usr/libexec/geoclue-yahoo
\end{Verbatim}
(Данный листинг был сокращен за счет удаления из него строк, относящихся к
потокам ядра, так как они никак не~относятся к обсуждаемой нами теме.)
\end{landscape}

Обратите внимание на третий столбец, показывающий имя контрольной группы,
которое systemd присваивает каждому процессу. Например, процесс +udev+
находится в группе +name=systemd:/systemd-1/sysinit.service+. В эту группу
помещаются процессы, запущенные службой +sysinit.service+, которая запускается
на ранней стадии загрузки.

Вы можете очень сильно упростить себе работу, если назначите для
вышеприведенной команды какой-нибудь простой и короткий псевдоним, например 
\begin{Verbatim}
alias psc='ps xawf -eo pid,user,cgroup,args'
\end{Verbatim}
---~теперь для получения исчерпывающей информации по процессам достаточно будет
нажать всего четыре клавиши.

Альтернативый способ получить ту же информацию~--- воспользоваться утилитой
+systemd-cgls+, входящей в комплект поставки systemd. Она отображает иерархию
контрольных групп в виде превдографической диаграммы-дерева: 

\begin{landscape}
\begin{Verbatim}[fontsize=\small]
$ systemd-cgls
+    2 [kthreadd]
[...]
+ 4281 [flush-8:0]
+ user
| \ lennart
|   \ 1
|     +  1495 pam: gdm-password
|     +  1521 gnome-session
|     +  1534 dbus-launch --sh-syntax --exit-with-session
|     +  1535 /bin/dbus-daemon --fork --print-pid 5 --print-address 7 --session
|     +  1603 /usr/libexec/gconfd-2
|     +  1612 /usr/libexec/gnome-settings-daemon
|     +  1615 /ushr/libexec/gvfsd
|     +  1621 metacity
|     +  1626 /usr/libexec//gvfs-fuse-daemon /home/lennart/.gvfs
|     +  1634 /usr/bin/pulseaudio --start --log-target=syslog
|     +  1635 gnome-panel
|     +  1638 nautilus
|     +  1640 /usr/libexec/polkit-gnome-authentication-agent-1
|     +  1641 /usr/bin/seapplet
|     +  1644 gnome-volume-control-applet
|     +  1645 /usr/libexec/bonobo-activation-server --ac-activate --ior-output-fd=24
|     +  1646 /usr/sbin/restorecond -u
|     +  1649 /usr/libexec/pulse/gconf-helper
|     +  1652 /usr/bin/devilspie
|     +  1662 nm-applet --sm-disable
|     +  1664 gnome-power-manager
|     +  1665 /usr/libexec/gdu-notification-daemon
|     +  1668 /usr/libexec/im-settings-daemon
|     +  1670 /usr/libexec/evolution/2.32/evolution-alarm-notify
|     +  1672 /usr/bin/python /usr/share/system-config-printer/applet.py
|     +  1674 /usr/lib64/deja-dup/deja-dup-monitor
|     +  1675 abrt-applet
|     +  1677 bluetooth-applet
|     +  1678 gpk-update-icon
|     +  1701 /usr/libexec/gvfs-gdu-volume-monitor
|     +  1707 /usr/bin/gnote --panel-applet --oaf-activate-iid=OAFIID:GnoteApplet_Factory --oaf-ior-fd=22
|     +  1725 /usr/libexec/clock-applet
|     +  1727 /usr/libexec/wnck-applet
|     +  1729 /usr/libexec/notification-area-applet
|     +  1759 gnome-screensaver
|     +  1780 /usr/libexec/gvfsd-trash --spawner :1.9 /org/gtk/gvfs/exec_spaw/0
|     +  1864 /usr/libexec/gvfs-afc-volume-monitor
|     +  1874 /usr/libexec/gconf-im-settings-daemon
|     +  1882 /usr/libexec/gvfs-gphoto2-volume-monitor
|     +  1903 /usr/libexec/gvfsd-burn --spawner :1.9 /org/gtk/gvfs/exec_spaw/1
|     +  1909 gnome-terminal
|     +  1913 gnome-pty-helper
|     +  1914 bash
|     +  1968 ssh-agent
|     +  1994 gpg-agent --daemon --write-env-file
|     +  2221 bash
|     +  2461 bash
|     +  4193 ssh tango
|     + 15113 bash
|     + 18679 /bin/sh /usr/lib64/firefox-3.6/run-mozilla.sh /usr/lib64/firefox-3.6/firefox
|     + 18741 /usr/lib64/firefox-3.6/firefox
|     + 27251 empathy
|     + 27262 /usr/libexec/mission-control-5
|     + 27265 /usr/libexec/telepathy-haze
|     + 27268 /usr/libexec/telepathy-logger
|     + 27270 /usr/libexec/dconf-service
|     + 27280 /usr/libexec/notification-daemon
|     + 27284 /usr/libexec/telepathy-gabble
|     + 27285 /usr/libexec/telepathy-salut
|     + 27297 /usr/libexec/geoclue-yahoo
|     + 28900 /usr/lib64/nspluginwrapper/npviewer.bin --plugin /usr/lib64/mozilla/plugins/libflashplayer.so --connection /org/wrapper/NSPlugins/libflashplayer.so/18741-6
|     + 29219 emacs systemd-for-admins-1.txt
|     + 29231 ssh tango
|     \ 29519 systemd-cgls
\ systemd-1
  + 1 /sbin/init
  + ntpd.service
  | \ 4112 /usr/sbin/ntpd -n -u ntp:ntp -g
  + systemd-logger.service
  | \ 1499 /lib/systemd/systemd-logger
  + accounts-daemon.service
  | \ 1496 /usr/libexec/accounts-daemon
  + rtkit-daemon.service
  | \ 1473 /usr/libexec/rtkit-daemon
  + console-kit-daemon.service
  | \ 1408 /usr/sbin/console-kit-daemon --no-daemon
  + prefdm.service
  | + 1376 /usr/sbin/gdm-binary -nodaemon
  | + 1391 /usr/libexec/gdm-simple-slave --display-id /org/gnome/DisplayManager/Display1 --force-active-vt
  | + 1394 /usr/bin/Xorg :0 -nr -verbose -auth /var/run/gdm/auth-for-gdm-f2KUOh/database -nolisten tcp vt1
  | + 1419 /usr/bin/dbus-launch --exit-with-session
  | \ 1511 /usr/bin/gnome-keyring-daemon --daemonize --login
  + getty@.service
  | + tty6
  | | \ 1346 /sbin/mingetty tty6
  | + tty4
  | | \ 1343 /sbin/mingetty tty4
  | + tty5
  | | \ 1342 /sbin/mingetty tty5
  | + tty3
  | | \ 1339 /sbin/mingetty tty3
  | \ tty2
  |   \ 1332 /sbin/mingetty tty2
  + abrtd.service
  | \ 1317 /usr/sbin/abrtd -d -s
  + crond.service
  | \ 1344 crond
  + sshd.service
  | \ 1362 /usr/sbin/sshd
  + sendmail.service
  | + 4094 sendmail: Queue runner@01:00:00 for /var/spool/clientmqueue
  | \ 4096 sendmail: accepting connections
  + haldaemon.service
  | + 1249 hald
  | + 1250 hald-runner
  | + 1273 hald-addon-input: Listening on /dev/input/event3 /dev/input/event9 /dev/input/event1 /dev/input/event7 /dev/input/event2 /dev/input/event0 /dev/input/event8
  | + 1275 /usr/libexec/hald-addon-rfkill-killswitch
  | + 1284 /usr/libexec/hald-addon-leds
  | + 1285 /usr/libexec/hald-addon-generic-backlight
  | \ 1287 /usr/libexec/hald-addon-acpi
  + irqbalance.service
  | \ 1210 irqbalance
  + avahi-daemon.service
  | + 1175 avahi-daemon: running [epsilon.local]
  + NetworkManager.service
  | + 1171 /usr/sbin/NetworkManager --no-daemon
  | \ 4028 /sbin/dhclient -d -4 -sf /usr/libexec/nm-dhcp-client.action -pf /var/run/dhclient-wlan0.pid -lf /var/lib/dhclient/dhclient-7d32a784-ede9-4cf6-9ee3-60edc0bce5ff-wlan0.lease -cf /var/run/nm-dhclient-wlan0.conf wlan0
  + rsyslog.service
  | \ 1193 /sbin/rsyslogd -c 4
  + mdmonitor.service
  | \ 1207 mdadm --monitor --scan -f --pid-file=/var/run/mdadm/mdadm.pid
  + cups.service
  | \ 1195 cupsd -C /etc/cups/cupsd.conf
  + auditd.service
  | + 1131 auditd
  | + 1133 /sbin/audispd
  | \ 1135 /usr/sbin/sedispatch
  + dbus.service
  | +  1096 /bin/dbus-daemon --system --address=systemd: --nofork --systemd-activation
  | +  1216 /usr/sbin/modem-manager
  | +  1219 /usr/libexec/polkit-1/polkitd
  | +  1242 /usr/sbin/wpa_supplicant -c /etc/wpa_supplicant/wpa_supplicant.conf -B -u -f /var/log/wpa_supplicant.log -P /var/run/wpa_supplicant.pid
  | +  1453 /usr/libexec/upowerd
  | +  1733 /usr/libexec/udisks-daemon
  | +  1747 udisks-daemon: polling /dev/sr0
  | \ 29509 /usr/libexec/packagekitd
  + dev-mqueue.mount
  + dev-hugepages.mount
  \ sysinit.service
    +   455 /sbin/udevd -d
    +  4016 /usr/sbin/bluetoothd --udev
    + 28188 /sbin/udevd -d
    \ 28191 /sbin/udevd -d
\end{Verbatim}
(Как и предыдущий, этот листинг был сокращен за счет удаления перечня потоков
ядра.)
\end{landscape}

Как видно из листинга, данная команда наглядно показывает принадлежность
процессов к их контрольным группам, а следовательно, и к службам, так как
systemd именует группы в соответствии с названиями служб. Например, из
приведенного листинга нетрудно понять, что служба системного аудита
+auditd.service+ порождает три отдельных процесса: +auditd+,
+audisp+ и +sedispatch+.

Внимательно посмотрев на листинг, можно заметить, что некоторые процессы
помещены в группу +/user/lennart/1+. Дело в том, что systemd занимается
отслежванием и группировкой не~только процессов, относящихся к системным
службам, но и процессов, запущенных в рамках пользовательских сеансов. В
последующих статьях мы обсудим этот вопрос более подробно. 

\section{HOW-TO: преобразование SysV init-скрипта в systemd service-файл}

Традиционно, службы Unix и Linux (демоны) запускаются через SysV init-скрипты.
Эти скрипты пишутся на языке Bourne Shell (+/bin/sh+), располагаются в
специальном каталоге (обычно +/etc/rc.d/init.d/+) и вызываются с одним из
стандартных параметров (+start+, +stop+, +reload+ и т.п.)~--- таким образом
указывается действие, которое необходимо прозвести над службой (запустить,
остановить, заставить перечитать конфигурацию). При запуске службы такой
скрипт, как правило, вызывает бинарник демона, который, в свою очередь,
форкается, порождая фоновый процесс (т.е. демонизируется). Заметим, что
shell-скрипты, как правило, отличается низкой скоростью работы, излишней
подробностью изложения и крайней хрупкостью. Читать их, из-за изобилия
всевозможного вспомогательного и дополнительного кода, чрезвычайно тяжело.
Впрочем, нельзя не~упомянуть, что эти скрипты являются очень гибким
инструментом (ведь, по сути, это всего лишь код, который можно модифицировать
как угодно). С другой стороны, многие задачи, возникающие при работе со
службами, бывает довольно тяжело решить средствами shell-скриптов. К таким
задачам относятся: огранизация параллельного исполнения, корректное
отслеживание процессов, конфигурирование различных параметров среды исполнения
процесса. systemd обеспечивает совместимость с init-скриптами, однако, с учетом
описанных выше их недостатков, более правильным решением будет использование
штатных service-файлов systemd для всех установленных в системе служб. Стоит
отметить что, в отличие от init-скриптов, которые часто приходится
модифицировать при переносе из одного дистриубтива в другой, один и тот же
service-файл будет работать в любом дистрибутиве, использующем systemd (а таких
дистрибутивов с каждым днем становится все больше и больше).  Далее мы вкратце
рассмотрим процесс преобразования SysV init-скрипта в service-файл systemd.
Вообще говоря, service-файл должен создаваться разработчиками каждого демона, и
включаться в комплект его поставки. Если вам удалось успешно создать
работоспособный service-файл для какого-либо демона, настоятельно рекомендуем
вам отправить этот файл разработчикам. Вопросы по полноценной интеграции
демонов с systemd, с максимальным использованием всех его возможностей, будут
рассмотрены в последующих статьях этого цикла, пока же ограничимся ссылкой на 
\href{http://0pointer.de/public/systemd-man/daemon.html}{страницу} официальной
документации. 

Итак, приступим. В качестве пример возьмем init-скрипт демона ABRT (Automatic
Bug Reporting Tool, службы, занимающейся сбором crash dump'ов). Исходный
скрипт (в варианте для дистрибутива Fedora) можно загрузить
\href{http://0pointer.de/public/abrtd}{здесь}.

Начнем с того, что прочитаем исходный скрипт (неожиданный ход, правда?) и
выделим полезную информацию из груды хлама. Практически у всех init-скриптов
большая часть кода является чисто вспомогательной, и мало чем отличается от
одного скрипта к другому. Как правило, при создании новых скриптов этот код
просто копируется из уже существующих (разработка в стиле copy-paste). Итак,
в исследуемом скрипте нас интересует следующая информация: 

\begin{itemize}
	\item Строка описания службы: <<Daemon to detect crashing apps>>. Как
		нетрудно заметить, комментарии в заголовке скрипта весьма
		пространны и описывают не~сколько саму службу, сколько
		скрипт, ее запускающий. service-файлы systemd также включают
		описание, но оно относится исключительно к службе, а не~к
		service-файлу. 
	\item LSB-заголовок\footnote{LSB-заголовок~--- определенная в
		\href{http://refspecs.freestandards.org/LSB_3.1.1/LSB-Core-generic/LSB-Core-generic/initscrcomconv.html}{Linux
		Standard Base} схема записи метаданных о службах в блоках
		комментариев соответствующих init-скриптов. Изначально эта
		схема была введена именно для того, чтобы стандартизировать
		init-скрипты во всех дистрибутивах. Однако разработчики
		многих дистрибутивов не~считают нужным точное исполнение
		требований LSB, и поэтому формы представления метаданных в
		различных дистрибутивах могут отличаться. Вследствие этого,
		при переносе init-скрипта из одного дистрибутива в другой,
		скрипт приходится модифицировать. Например, демон пересылки
		почты при описании зависимостей может именоваться
		+MTA+ или +smtpdaemon+ (Fedora), +smtp+
		(openSUSE), +mail-transport-agent+ (Debian и Ubuntu),
		+mail-transfer-agent+. Таким образом,
		стандарт LSB не~справляется с поставленной задачей},
		содержащий информацию о зависимостях.  systemd, базирующийся
		на идеях socket-активации, обычно не~требует явного описания
		зависимостей (либо требует самого минимального описания).
		Заметим, что основополагающие принципы systemd, включая
		socket-активацию, рассмотрены в статье
		\href{http://0pointer.de/blog/projects/systemd.html}{Rethinking
		PID 1}, в которой systemd был впервые представлен широкой
		публике. Ее русский перевод можно прочитать здесь:
		\href{http://tux-the-penguin.blogspot.com/2010/09/systemd.html}{часть~1},
		\href{http://tux-the-penguin.blogspot.com/2010/09/systemd-ii.html}{часть~2}.
		Возвращаясь к нашему примеру: в данном случае ценной
		информацией о зависимостях является только строка
		+Required-Start: $syslog+, сообщающая, что для работы
		abrtd требуется демон системного лога.  Информация о второй
		зависимости, +$local_fs+, является избыточной, так как
		systemd приступает к запуску служб уже после того, как все
		файловые системы готовы для работы. 
	\item Также, LSB-заголовок сообщает, что данная служба должна быть
		запущена на уровнях исполнения (runlevels) 3 (консольный
		многопользовательский) и 5 (графический
		многопользовательской).
	\item Исполняемый бинарник демона называется +/usr/sbin/abrtd+. 
\end{itemize}

Вот и вся полезная информация. Все остальное содержимое 115-строчного скрипта
является чисто вспомогательным кодом: операции синхронизации и упорядочивания
запуска (код, относящийся к lock-файлам), вывод информационных сообщений
(команды +echo+), разбор входных параметров (монструозный блок
+case+).

На основе приведенной выше полезной информации, мы можем написать следующий
service-файл: 
\begin{Verbatim}
[Unit]
Description=Daemon to detect crashing apps
After=syslog.target

[Service]
ExecStart=/usr/sbin/abrtd
Type=forking

[Install]
WantedBy=multi-user.target
\end{Verbatim}

Рассмотрим этот файл поподробнее.

Секция +[Unit]+ содежит самую общую информацию о службе. Не будем
забывать, что systemd управляет не~только службами, но и многими другими
объектами, в частности, устройствами, точками монтирования, таймерами и т.п.
Общее наименование всех этих объектов~--- юнит (unit). Одноименная секция
конфигурационного файла определяет наиболее общие свойства, которые могут
быть присущи любому юниту. В нашем случае это, во-первых, строка описания, и
во-вторых, указание, что данный юнит рекомендуется активировать после запуска
демона системного лога\footnote{Строго говоря, эту зависимость здесь
указывать не~нужно~--- в системах, в которых демон системного лога активируется
через сокет, эта зависимость является избыточной. Современные реализации
демона системного лога (например, rsyslog начиная с пятой версии)
поддерживают активацию через сокет. В системах, использующих такие
реализации, явное указание +After=syslog.target+ будет избыточным, так
как соответствующая функциональность поддерживается автоматически. Однако,
эту строчку стоит все-таки указать для обеспечения совместимости с системами,
использующими устаревшие реализации демона системного лога}. Эта информация,
как мы помним, была указана в LSB-заголовке исходного init-скрипта. В нашем
конфигурационном файле мы указываем зависимость от демона системного лога при
помощи директивы +After+, ссылающейся на юнит +syslog.taget+. Это
специальный юнит, позволяющий ссылаться на любую реализацию демона системного
лога, независимо от используемой программы (например, rsyslog или syslog-ng)
и типа активации (как обычной службы или через log-сокет). Подробнее о таки
специальных юнитах можно почитать
\href{http://0pointer.de/public/systemd-man/systemd.special.html}{страницу}
официальной документации. Обратите внимание, что директива +After+, в
отсутствие директивы +Requires+, задает лишь порядок загрузки, но
не~задает жесткой зависимости. То есть, если при загрузке конфигурация
systemd будет предписывать запуск как демона системного лога, так и abrtd, то
сначала будет запущен демон системного лога, и только потом abrtd. Если же
конфигурация не~будет содержать явного указания запустить демон системного
лога, он не~будет запущен даже при запуске abrtd. И это поведение нас
полностью устраивает, так как abrtd прекрасно может обходиться и без демона
системного лога. В противном случае, мы могли бы воспользоваться директивой
+Requires+, задающей жесткую зависимость между юнитами.

Следующая секция, +[Service]+, содержит информацию о службе. Сюда включаются
настройки, относящие именно к службам, но не~к другим типа юнитов. В нашем
случае, таких настроек две: +ExecStart+ определяет расположение бинарника
демона и аргументы, с которыми он будет вызван (в нашем случае они
отсутствуют), и +Type+, позволяющая задать метод, по которому systemd определит
окончание периода запуска службы. Традиционный для Unix метод демонизации
процесса, когда исходный процесс форкается, порождая демона, после чего
завершается, описывается типом +forking+ (как в нашем случае). Таким образом,
systemd считает службу запущенной с момента завершения работы исходного
процесса, и рассматривает в качестве основного процесса этой службы
порожденный им процесс-демон.

И наконец, третья секция, +[Install]+. Она содержит рекомендации по
установке конкретного юнита, указывающие, в каких ситуациях он должен быть
активирован.  В нашем случае, служба abrtd запускается при активации юнита
+multi-user.target+. Это специальный юнит, примерно соответствующий роли
третьего уровня исполнения классического SysV\footnote{В том контексте, в
котором он используется в большинстве дистрибутивов семейства Red Hat, а
именно, многопользовательский режим без запуска графической оболочки}.
Директива +WantedBy+ никак не~влияет на уже работающую службу, но она
играет важную роль при выполнении команды systemctl enable, задавая, в каких
условиях должен активироваться устанавливаемый юнит. В нашем примере, служба
abrtd будет активироваться при переходе в состояние +multi-user.target+,
т.е., при каждой нормальной загрузке\footnote{Обратите внимание, что режим
графической загрузки в systemd (+graphical.target+, аналог runlevel 5
в SysV) является надстройкой над режимом многопользовательской консольной
загрузки (+multi-user.target+, аналог runlevel 3 в SysV). Таким
образом, все службы, запускаемые в режиме +multi-user.target+, будут
также запускаться и в режиме +graphical.target+} (к <<ненормальным>>
можно отнести, например, загрузки в режиме +emergency.target+, который
является аналогом первого уровня исполнения В классической SysV).

Вот и все. Мы получили минимальный рабочий service-файл systemd. Чтобы
проверить его работоспособность, скопируем его в
+/etc/systemd/system/abrtd.service+, после чего командой
+systemctl daemon-reload+ уведомим systemd об изменении конфигурации.
Теперь нам остается только запустить нашу службу:
+systemctl start abrtd.service+.  Проверить состояние службы можно
командой +systemctl status abrtd.service+, а чтобы остановить ее, нужно
скомандовать +systemctl stop abrtd.service+. И наконец, команда
+systemctl enable abrtd.service+ выполнит установку service-файла,
обеспечив его активацию при каждой загрузке (аналог +chkconfig abrtd on+
в классическом SysV).

Приведенный выше service-файл является практический точным переводом
исходного init-скрипта, и он никак не~использует широкий спектр возможностей,
предоставляемых systemd. Ниже приведен немного улучшенный вариант: 

\begin{Verbatim}
[Unit]
Description=ABRT Automated Bug Reporting Tool
After=syslog.target

[Service]
Type=dbus
BusName=com.redhat.abrt
ExecStart=/usr/sbin/abrtd -d -s

[Install]
WantedBy=multi-user.target
\end{Verbatim}

Чем же новый вариант отличается от предыдущего? Ну, прежде всего, мы уточнили
описание службы. Однако, ключевым изменением является замена +Type+ с +forking+
на +dbus+ и связанные с ней изменения: добавление имени службы в шине D-Bus
(директива +BusName+) и задание полонительных аргументов abrtd <<+-d -s+>>. Но
зачем вообще нужна эта замена? Каков ее практический смысл? Чтобы ответить на
этот вопрос, мы снова возращаемся к демонизации. В ходе этой операции, процесс
дважды форкается и отключается от всех терминалов. Это очень удобно при запуске
демона через скрипт, но в случае использования таких продвинутых систем
инициализации, как systemd, такое поведение не~дает никаких преимуществ, но
вызывает неоправданные задержки. Даже если мы оставим в стороне вопрос скорости
загрузки, останется такой важный аспект, как отслеживание состояния служб.
systemd решает и эту задачу, контролируя работу службы и при необходимости
реагируя на различные события. Например, при неожиданном падении основного
процесса службы, systemd должен зарегистрировать идентификатор и код выхода
процесса, также, в зависимости от настроек, он может попытаться перезапустить
службу, либо активировать какой-либо заранее заданный юнит. Операция
демонизации несколько затрудняет решение этих задач, так как обычно довольно
сложно найти связь демонизированного процесса с исходным (собственно, смысл
демонизации как раз и сводится к уничтожению этой связи) и, сооветственно, для
systemd сложнее определить, какой из порожденных в рамках данной службы
процессов является основным. Чтобы упростить для него решение этой задачи, мы и
воспользовались типом запуска +dbus+. Он подходит для всех служб, которые в
конце процесса инициализации регистрируют свое имя на шине D-Bus\footnote{В
настоящее время практически все службы дистрибутива Fedora после запуска
регистрируется на шине D-Bus}. ABRTd относится к ним.  С новыми настройками,
systemd запусит процесс abrtd, который уже не~будет форкаться (согласно
указанным нами ключам <<+-d -s+>>), и в качестве момента окончания периода
запуска данной службы systemd будет рассматривать момент регистрации имени
+com.redhat.abrt+ на шине D-Bus. В этом случае основным для данной службы будет
считаться процесс, непосредственно порожденный systemd.  Таким образом, systemd
располагает удобным методом для определения момента окончания запуска службы, а
также может легко отслеживать ее состояние. 

Собственно, это все, что нужно было сделать. Мы получили простой
конфигурационный файл, в 10 строчках которого содержится больше полезной
информации, чем в 115 строках исходного init-скрипта. Добавляя в наш файл по
одной строчке, мы можем использовать различные полезные функции systemd,
создание аналога которых в традиционном init-скрипте потребовало бы
значительных усилий. Например, добавив строку +Restart=restart-always+, мы
приказываем systemd автоматически перезапускать службу после каждого ее
падения. Или, например, добавив +OOMScoreAdjust=-500+, мы попросим ядро сберечь
эту службу, даже если OOM Killer выйдет на тропу войны. А если мы добавим
строчку +CPUSchedulingPolicy=idle+, процесс abrtd будет работать только в те
моменты, когда система больше ничем не~занята, что позволит не создавать помех
для процессов, активно использующих CPU. 

За более подробным описанием всех опций настройки, вы можете обратиться к
страницам рукводства
\href{http://0pointer.de/public/systemd-man/systemd.unit.html}{systemd.unit},
\href{http://0pointer.de/public/systemd-man/systemd.service.html}{systemd.service},
\href{http://0pointer.de/public/systemd-man/systemd.exec.html}{systemd.exec}. Полный
список доступных страниц можно просмотреть
\href{http://0pointer.de/public/systemd-man/}{здесь}.

Конечно, отнюдь не~все init-скрипты так же легко преобразовать в
service-файлы. Но, к счастью, <<проблемных>> скриптов не~так уж и много. 

\section{Убить демона}

Убить системного демона нетрудно, правда? Или\ldots все не~так просто?

Если ваш демон функционирует как один процесс, все действительно очень просто.
Вы командуете +killall rsyslogd+, и демон системного лога останавливается.
Впрочем, этот метод не~вполне корректен, так как он действует не~только на
самого демона, но и на другие процессы с тем же именем. Иногда подобное
поведение может привести к неприятным последствиям. Более правильным будет
использование pid-файла: +kill \$(cat /var/run/syslogd.pid)+. Вот, вроде
бы, и все, что вам нужно\ldots Или мы упускаем еще что-то?

Действительно, мы забываем про одну простую вещь: существуют службы, такие, как
Apache, crond, atd, которые по роду служебной дейятельности должны запускать
дочерние процессы. Это могут быть совершенно посторонние, указаанные
пользователем программы (например, задачи cron/at, CGI-скрипты) или полноценные
серверные процессы (например, Apache workers). Когда вы убиваете основной
процесс, он может остановить все дочерние процессы. А может и не~остановить. В
самом деле, если служба функционирует в штатном режиме, ее обычно останавливают
специальной командой stop. К прямому вызову kill администратор, как правило,
прибегает только в аварийной ситуации, когда служба работает неправильно и
может не~среагировать на стандартную команду остановки. Таким образом, убив,
например, основной сервер Apache, вы можете получить от него в наследство
работающие CGI-скрипты, причем их родителем автоматически станет PID 1 (init),
так что установить их происхождение будет не~так-то просто. 

\href{http://www.freedesktop.org/wiki/Software/systemd}{systemd} спешит к нам
на помощь. Команда +systemctl kill+ позволит отправить сигнал всем
процессам, порожденным в рамках данной службы. Например: 

\begin{Verbatim}
# systemctl kill crond.service
\end{Verbatim}

Вы можете быть уверены, что всем процессам службы cron будет отправлен сигнал
+SIGTERM+. Разумеется, можно отправить и любой другой сигнал. Скажем, если ваши
дела совсем уж плохи, вы можете воспользоваться и +SIGKILL+: 

\begin{Verbatim}
# systemctl kill -s SIGKILL crond.service
\end{Verbatim}

После ввода этой команды, служба cron будет жестоко убита вместе со всеми ее
дочерними процессами, вне зависимости от того, сколько раз она форкалась, и
как бы она ни пыталась сбежать из-под нашего контроля при помощи двойного
форка или
\href{http://ru.wikipedia.org/wiki/Fork-%D0%B1%D0%BE%D0%BC%D0%B1%D0%B0}{форк-бомбардировки}%
\footnote{Прим. перев.: стоит особо отметить, что использование контрольных
групп не~только упрощает процесс уничтожения форк-бомб, но и значительно
уменьшает ущерб от работающей форк-бомбы. Так как systemd автоматически
помещает каждую службу и каждый пользовательский сеанс в свою контрольную
группу по ресурсу процессорного времени, запуск форк-бомбы одним
пользователем или службой не~создаст значительных проблем с отзывчивостью
системы у других пользователей и служб. Таким образом, в качестве основной
угрозы форк-бомбардировки остаются лишь возможности исчерпания памяти и
идентификаторов процессов (PID)}.

В некоторый случах возникает необходимость отправить сигнал именно основному
процессу службы. Например, используя +SIGHUP+, мы можем заставить демона
перечитать файлы конфигурации. Разумеется, вспомогательным процессам
передавать HUP в этом случае совершенно необязательно. Для решения этой
задачи вполbb неплохо подойдет и классический метод с pid-файлом, однако у
systemd и на этот случай есть просто решение, избавляющее вас от
необходимости искать нужный файл: 

\begin{Verbatim}
# systemctl kill -s HUP --kill-who=main crond.service
\end{Verbatim}

Итак, что же принципиально новое привносит systemd в рутинный процесс
убийства демона? Прежде всего: впервые в истории Linux представлен способ
принудительной остановки службы, не~зависящий от того, насколько
добросовестно основной процесс службы выполняет свои обязательства по
остановке дочерних процессов. Как уже упоминалось выше, необходимость
отправить процессу +SIGTERM+ или +SIGKILL+ обычно возникает именно
в нештатной ситуации, когда вы уже не~можете быть уверены, что демон
корректно исполнит все свои обязанности.

После прочтения сказанного выше у вас может возникнуть вопрос: в чем разница
между +systemctl kill+ и +systemctl stop+? Отличие состоит в том,
что +kill+ просто отправляет сигнал заданному процессу, в то время как
stop действует по <<официально>> определенному методу, вызывая команду,
определенную в параметре +ExecStop+ конфигурации службы. Обычно команды
stop бывает вполне достаточно для остановки службы, и к +kill+
приходится прибегать только в крайних случаях, например, когда служба
<<зависла>> и не~реагирует на команды.

Кстати говоря, при использовании параметра <<+-s+>>, вы можете указывать
названия сигналов как с префиксом SIG, так и без него~--- оба варианта будут
работать.

В завершение стоит сказать, что для нас весьма интересным и неожиданным
оказался тот факт, что до появления systemd в Linux просто не~существовало
инструментов, позволяющих корректно отправить сигнал службе в целом, а
не~отдельному процессу.

\end{document}
